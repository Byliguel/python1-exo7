\documentclass[12pt,class=report,crop=false]{standalone}
\usepackage[screen]{../python}


\pagestyle{empty}

\begin{document}


%====================================================================
\chapitre{Jeu de la vie}
%====================================================================

\textbf{Règle du jeu.}


\begin{itemize}
  \item Pour une case vide au jour $j$ et ayant exactement $3$ cellules voisines : une cellule naît au jour $j+1$.

\myfigure{0.5}{
  \tikzinput{fig-vie-0a}
}

 \bigskip

  \item Pour une case contenant une cellule au jour $j$, ayant soit $2$ ou soit $3$ cellules voisines : alors la cellule continue de vivre.
  Dans les autres cas la cellule meurt (avec $0$ ou $1$, elle meurt d'isolement, avec plus de $4$ voisins, elle meurt de surpopulation !).
  
\myfigure{0.5}{
  \tikzinput{fig-vie-0b}
}  
  
  
\end{itemize}

\newpage
 
Le \og{}clignotant\fg{} :

\myfigure{0.5}{
  \tikzinput{fig-vie-0c}
} 
 
 

 \newpage
 
\textbf{Modélisation}

\myfigure{0.6}{
  \tikzinput{fig-vie-1}
} 

\begin{itemize}
  \item \ci{tableau = [[0 for j in range(p)] for i in range(n)]}
    
  \item \ci{tableau[i][j] = 1} 
  
  \item Affichage non graphique :
 \begin{center}
\ci{00000000}\\
\ci{00000000}\\
\ci{00111000}\\
\ci{00000000}\\
\ci{00000000}
\end{center} 
  
  \end{itemize}





\end{document}
