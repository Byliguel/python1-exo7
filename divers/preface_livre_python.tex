
\pagestyle{empty}\thispagestyle{empty}
\vspace*{\fill}
\vspace*{5ex}
\begin{center}
\fontsize{52}{52}\selectfont
\textsc{python au lycée}

\vspace*{1ex}

\fontsize{32}{32}\selectfont
\textsc{arnaud bodin}
  %\hfil  
 % \includegraphics[width = 6cm]{logoScratchAuCollege}
\end{center}
\vfill
\begin{center}
\huge
\textsc{algorithmes \  et \  programmation}
\end{center}
\begin{center}
\LogoExoSept{2}
\end{center}
\clearemptydoublepage
%\clearpage

\thispagestyle{empty}

\vspace*{\fill}
\section*{Python au lycée}

%---------------------------
{\large\textbf{C'est parti !}}

Tout le monde utilise un ordinateur, mais c'est une autre chose de le piloter ! Tu vas apprendre ici les bases de la programmation.  L'objectif de ce livre est double :  approfondir les mathématiques à travers l'informatique et maîtriser la programmation en s'aidant des mathématiques. 

\bigskip

%---------------------------
{\large\textbf{Python}}

Choisir un langage de programmation pour débuter est délicat. Il faut un langage avec une prise en main facile, bien documenté, avec une grande communauté d'utilisateurs. Python possède toutes ces qualités et davantage encore. Il est moderne, puissant et très utilisé, y compris par les programmeurs professionnels. 

Malgré toutes ces qualités, débuter la programmation (avec Python ou un autre langage) est difficile. Le mieux est d'avoir déjà une expérience du code, à l'aide de \emph{Scratch} par exemple. Il reste quand même une grande marche à gravir et ce livre est là pour t'accompagner.

\bigskip

%---------------------------
{\large\textbf{Objectif}}

Bien maîtriser Python te permettra d'apprendre facilement les autres langages. Surtout le langage n'est pas le plus important, l'essentiel ce sont les algorithmes. Les algorithmes sont comme des recettes de cuisine, il faut suivre pas à pas les instructions et ce qui compte, c'est le résultat final et non le langage avec lequel a été écrite la recette. Ce livre n'est donc ni un manuel complet de Python, ni un cours d'informatique, il ne s'agit pas non plus d'utiliser Python comme une super-calculatrice.

Le but est de découvrir des algorithmes, d'apprendre la programmation pas à pas à travers des activités mathématiques/informatiques. Cela te permettra de mettre en pratique des mathématiques avec ici la volonté de se limiter aux connaissances acquises au niveau seconde.

\bigskip

%---------------------------
{\large\textbf{Mathématiques pour l'informatique}}\\
{\large\textbf{Informatique pour les mathématiques}}

Comme les ordinateurs ne manipulent que des nombres, les mathématiques sont indispensables pour communiquer avec eux. Un exemple est l'écriture binaire qui utilise les puissances de 2, la division euclidienne\ldots{} Un autre exemple est l'affichage graphique à l'écran qui nécessite de bien maîtriser les  coordonnées $(x,y)$, la trigonométrie\ldots

L'informatique accompagne à merveille les mathématiques ! L'ordinateur devient indispensable pour manipuler de très grands nombres ou bien tester des conjectures sur de nombreux cas. Tu découvriras dans ce livre des fractales, des L-systèmes, des arbres browniens\ldots{} et la beauté de phénomènes mathématiques complexes.

\vspace*{\fill}



%\newpage
\cleardoublepage
\thispagestyle{empty}
\addtocontents{toc}{\protect\setcounter{tocdepth}{0}}
\tableofcontents


\newpage

\section*{Résumé des activités}


\newcommand{\titreactivite}[1]{{\large\textbf{#1}}\nopagebreak}
\newcommand{\descriptionactivite}[1]{%
\smallskip\hfill
\begin{minipage}{0.95\textwidth}#1\end{minipage}\medskip\smallskip}

%%%%%%%%%%%%%%%%%%%%%%%%%%%%%%%%%%%%%%%%%%%%%%%%%%%%
\titreactivite{Premiers pas}

\descriptionactivite{Lance-toi dans la programmation ! Dans cette toute première activité, tu vas apprendre à manipuler des nombres, des variables et tu vas coder tes premières boucles avec \Python.}

%%%%%%%%%%%%%%%%%%%%%%%%%%%%%%%%%%%%%%%%%%%%%%%%%%%%
\titreactivite{Tortue (Scratch avec Python)}

\descriptionactivite{Le module \ci{turtle}
 permet de tracer facilement des dessins en \Python{}. Il s'agit de commander une tortue à l'aide d'instructions simples comme \og{}avancer\fg{}, \og{}tourner\fg{}\ldots{} C'est le même principe qu'avec \emph{Scratch}, avec toutefois des différences : tu ne déplaces plus des blocs, mais tu écris les instructions ; et en plus les instructions sont en anglais !}

%%%%%%%%%%%%%%%%%%%%%%%%%%%%%%%%%%%%%%%%%%%%%%%%%%%%
\titreactivite{Si ... alors ...}

\descriptionactivite{L'ordinateur peut réagir en fonction d'une situation. 
Si une condition est remplie il agit d'une certaine façon, 
sinon il fait autre chose.}

%%%%%%%%%%%%%%%%%%%%%%%%%%%%%%%%%%%%%%%%%%%%%%%%%%%%
\titreactivite{Fonctions}

\descriptionactivite{\'Ecrire une fonction, c'est la façon la plus simple de regrouper du code pour une tâche bien particulière, dans le but de l'exécuter une ou plusieurs fois par la suite.}

%%%%%%%%%%%%%%%%%%%%%%%%%%%%%%%%%%%%%%%%%%%%%%%%%%%%
\titreactivite{Arithmétique -- Boucle tant que -- I}

\descriptionactivite{Les activités de cette fiche sont centrées sur l'arithmétique : division euclidienne, nombres premiers\ldots{} C'est l'occasion d'utiliser intensivement la boucle \og{}tant que\fg{}.}

%%%%%%%%%%%%%%%%%%%%%%%%%%%%%%%%%%%%%%%%%%%%%%%%%%%%
\titreactivite{Chaînes de caractères -- Analyse d’un texte}

\descriptionactivite{Tu vas faire quelques activités amusantes en manipulant les chaînes de caractères.}

%%%%%%%%%%%%%%%%%%%%%%%%%%%%%%%%%%%%%%%%%%%%%%%%%%%%
\titreactivite{Listes I}

\descriptionactivite{Une liste est une façon de regrouper des éléments en un seul objet. Après avoir défini une liste, on peut récupérer un par un chaque élément de la liste, mais aussi en ajouter de nouveaux\ldots}

%%%%%%%%%%%%%%%%%%%%%%%%%%%%%%%%%%%%%%%%%%%%%%%%%%%%
\titreactivite{Statistique -- Visualisation de données}

\descriptionactivite{C'est bien de savoir calculer le minimum, le maximum, la moyenne, les quartiles d'une série. C'est mieux de les visualiser tous sur un même graphique !}

%%%%%%%%%%%%%%%%%%%%%%%%%%%%%%%%%%%%%%%%%%%%%%%%%%%%
\titreactivite{Fichiers}

\descriptionactivite{Tu vas apprendre à lire et à écrire des données dans des fichiers.}

%%%%%%%%%%%%%%%%%%%%%%%%%%%%%%%%%%%%%%%%%%%%%%%%%%%%
\titreactivite{Arithmétique -- Boucle tant que -- II}

\descriptionactivite{On approfondit notre étude des nombres avec la boucle \og{}tant que\fg{}. Pour cette fiche tu as besoin d'une fonction \ci{est_premier()} construite dans la fiche
\og{}Arithmétique -- Boucle tant que -- I\fg{}.}

%%%%%%%%%%%%%%%%%%%%%%%%%%%%%%%%%%%%%%%%%%%%%%%%%%%%
\titreactivite{Binaire I}

\descriptionactivite{Les ordinateurs transforment toutes les données en nombres et manipulent uniquement ces nombres. Ces nombres sont stockés sous la forme de listes de $0$ et de $1$. C'est l'écriture binaire des nombres !
Pour mieux comprendre l'écriture binaire, tu vas d'abord mieux comprendre l'écriture décimale.
}

%%%%%%%%%%%%%%%%%%%%%%%%%%%%%%%%%%%%%%%%%%%%%%%%%%%%
\titreactivite{Listes II}

\descriptionactivite{Les listes sont tellement utiles qu'il faut savoir les manipuler de façon simple et efficace. C'est le but de cette fiche !}

%%%%%%%%%%%%%%%%%%%%%%%%%%%%%%%%%%%%%%%%%%%%%%%%%%%%
\titreactivite{Binaire II}

\descriptionactivite{On continue notre exploration du monde des $0$ et des $1$.}

%%%%%%%%%%%%%%%%%%%%%%%%%%%%%%%%%%%%%%%%%%%%%%%%%%%%
\titreactivite{Probabilités – Paradoxe de Parrondo}

\descriptionactivite{Tu vas programmer deux jeux simples. Lorsque tu joues à ces jeux, tu as plus de chances de perdre que de gagner. Pourtant lorsque tu joues aux deux jeux en même temps, tu as plus de chances de gagner que de perdre ! C'est une situation paradoxale.}

%%%%%%%%%%%%%%%%%%%%%%%%%%%%%%%%%%%%%%%%%%%%%%%%%%%%
\titreactivite{Chercher et remplacer}

\descriptionactivite{Chercher et remplacer sont deux tâches très fréquentes. Savoir les utiliser et comprendre comment elles fonctionnent te permettra d'être plus efficace.}

%%%%%%%%%%%%%%%%%%%%%%%%%%%%%%%%%%%%%%%%%%%%%%%%%%%%
\titreactivite{Calculatrice polonaise -- Piles}

\descriptionactivite{Tu vas programmer ta propre calculatrice ! Pour cela tu vas découvrir une nouvelle notation pour les formules et aussi découvrir ce qu'est une \og{}pile\fg{} en informatique.}

%%%%%%%%%%%%%%%%%%%%%%%%%%%%%%%%%%%%%%%%%%%%%%%%%%%%
\titreactivite{Visualiseur de texte -- Markdown}

\descriptionactivite{Tu vas programmer un traitement de texte tout simple qui affiche proprement des paragraphes et met en évidence les mots en gras et en italiques.}

%%%%%%%%%%%%%%%%%%%%%%%%%%%%%%%%%%%%%%%%%%%%%%%%%%%%
\titreactivite{L-système}

\descriptionactivite{Les L-systèmes offrent une façon très simple de coder des phénomènes complexes. À partir d'un mot initial et d'opérations de remplacement, on arrive à des mots compliqués. Lorsque l'on \og{}dessine\fg{} ces mots, on obtient de superbes figures fractales. Le \og{}L\fg{} vient du botaniste A. Lindenmayer qui a inventé les L-systèmes afin de modéliser les plantes.
}

%%%%%%%%%%%%%%%%%%%%%%%%%%%%%%%%%%%%%%%%%%%%%%%%%%%%
\titreactivite{Images dynamiques}

\descriptionactivite{Nous allons déformer des images. En répétant ces déformations, les images deviennent brouillées. Mais par miracle au bout d'un certain nombre de répétitions l'image de départ réapparaît !}


%%%%%%%%%%%%%%%%%%%%%%%%%%%%%%%%%%%%%%%%%%%%%%%%%%%%
\titreactivite{Jeu de la vie}

\descriptionactivite{Le \emph{jeu de la vie} est un modèle simple de l'évolution d'une population de cellules qui naissent et meurent au cours du temps. Le \og{}jeu\fg{} consiste à trouver des configurations initiales qui donnent des évolutions intéressantes : certains groupes de cellules disparaissent, d'autres se stabilisent, certains se déplacent\ldots}

%%%%%%%%%%%%%%%%%%%%%%%%%%%%%%%%%%%%%%%%%%%%%%%%%%%%
\titreactivite{Graphes et combinatoire de Ramsey}

\descriptionactivite{Tu vas voir qu'un problème tout simple, qui concerne les relations entre seulement six personnes, va demander énormément de calculs pour être résolu.}

%%%%%%%%%%%%%%%%%%%%%%%%%%%%%%%%%%%%%%%%%%%%%%%%%%%%
\titreactivite{Bitcoin}

\descriptionactivite{Le \emph{bitcoin} est une monnaie dématérialisée et décentralisée. Elle repose sur deux principes informatiques : la cryptographie à clé publique et la preuve de travail. Pour comprendre ce second principe, tu vas créer un modèle simple de \emph{bitcoin}.}

%%%%%%%%%%%%%%%%%%%%%%%%%%%%%%%%%%%%%%%%%%%%%%%%%%%%
\titreactivite{Constructions aléatoires}

\descriptionactivite{Tu vas programmer deux méthodes pour construire des figures qui ressemblent à des algues ou des coraux. Chaque figure est formée de petits blocs lancés au hasard et qui se collent les uns aux autres.}




