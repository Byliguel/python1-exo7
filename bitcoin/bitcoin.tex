\documentclass[11pt,class=report,crop=false]{standalone}
\usepackage[screen]{../python}

\begin{document}


%====================================================================
\chapitre{Bitcoin}
%====================================================================

\objectifs{Le \emph{bitcoin} est une monnaie dématérialisée et décentralisée. Elle repose sur deux principes informatiques : la cryptographie à clé publique et la preuve de travail. Pour comprendre ce second principe, tu vas créer un modèle simple de \emph{bitcoin}.}

\index{bitcoin@\emph{bitcoin}}


\insertvideo{lcVC2fqByEo}{Bitcoin - partie 1 - principe général}

\insertvideo{6yZAHoPGGbw}{Bitcoin - partie 2 - fonction de hachage}

\insertvideo{FPh9huqhJJg}{Bitcoin - partie 3 - preuve de travail}

\insertvideo{XfVpekxADyM}{Bitcoin - partie 4 - tes bitcoins}



%%%%%%%%%%%%%%%%%%%%%%%%%%%%%%%%%%%%%%%%%%%%%%%%%%%%%%%%%%%%%%%%
% Activité 1
%%%%%%%%%%%%%%%%%%%%%%%%%%%%%%%%%%%%%%%%%%%%%%%%%%%%%%%%%%%%%%%%

\begin{activite}[Preuve de travail]

\objectifs{Objectifs : comprendre ce qu'est une preuve de travail sur un modèle simple. Cette activité est indépendante du reste de la fiche.
L'idée est de trouver un problème difficile à résoudre mais facile à vérifier. Comme les sudokus par exemple : il suffit de dix secondes pour vérifier qu'une grille est remplie correctement, par contre il a fallu plus de dix minutes pour le résoudre.}

\index{preuve de travail}

Le problème mathématique est le suivant : on te donne un nombre premier $p$  et un entier $y$ ; tu dois trouver un entier $x$ tel que :
$$x^2 = y \pmod{p}$$
Autrement dit $x$ est une racine carrée de $y$ modulo $p$.
Attention, il n'existe pas toujours de solution pour $x$. 

\textbf{Exemples.}
\begin{itemize}
  \item Pour $p=13$ et $y = 10$, alors une solution est $x = 6$.
  Effet $x^2 = 6^2 = 36$. Et $36 = 2\times 13 + 10$ donc $x^2 = 10 \pmod{13}$.
  
  \item La solution n'est pas forcément unique. Par exemple, vérifie que $x=7$ est aussi solution.
  
  \item Il n'y a pas toujours de solution, par exemple pour $p=13$ et $y=5$, aucun entier $x$ n'est solution.
  
  \item Exercice : pour $p = 13$, trouve à la main deux solutions $x$ au problème $x^2 = 9 \pmod{13}$ ; trouve une solution $x$ au problème $x^2 = 3 \pmod{13}$.
  
\end{itemize}


L'entier $p$ est fixé pour toute l'activité. 
Pour les exemples faciles on prendra $p = 101$, pour les exemples moyens $p = 15\,486\,869$ et pour les exemples difficiles $p = 2\,276\,856\,017$. Plus l'entier $p$ est grand, plus la preuve de travail est difficile à obtenir.

\begin{enumerate}
  \item \textbf{Vérification (facile).} 
  \'Ecris une fonction \ci{verification(x,y)} qui renvoie \og{}vrai\fg{} si $x$ est bien solution du problème $x^2 = y \pmod{p}$ et \og{}faux\fg{} sinon.
  
  Vérifie que $x = 6\,543\,210$ est solution lorsque $y = 8\,371\,779$ et $p = 15\,486\,869$. Affiche le temps de calcul nécessaire à cette vérification. (Voir le cours plus bas.)
  
  \item \textbf{Recherche de solution (difficile).} Pour trouver une solution $x$, il n'y a pas vraiment d'autres choix pour nous que de tester tous les $x$ en commençant par $x=0$, $x=1$\ldots{}
Programme une fonction \ci{racine(y)} qui renvoie une solution $x$ du problème pour $y$ donné (ou \ci{None} s'il n'y a pas de solution).

  \begin{itemize}
    \item Pour $p = 101$ et $y=17$, trouve $x$ tel que $x^2 = y \pmod{p}$. 
    \item Pour $p = 15\,486\,869$ et $y = 8\,371\,779$, tu dois retrouver le $x$ de la première question. Combien de temps a pris la recherche ? 
    
    \item Pour $p = 15\,486\,869$ et $y = 13\,017\,204$, trouve $x$ tel que $x^2 = y \pmod{p}$.    
   \end{itemize}


\medskip

\emph{Conclusion : nous avons trouvé un problème difficile à résoudre, mais pour lequel il est facile de vérifier qu'une solution donnée convient. Pour des valeurs de $p$ plus grandes, la recherche d'une solution $x$ peut être beaucoup trop longue et ne pas aboutir. Nous allons voir comment on peut ajuster la difficulté du problème.}


	\item Au lieu de chercher une solution exacte à notre problème $x^2 = y \pmod{p}$, qui est équivalent à $x^2-y \pmod{p} = 0$. On cherche une solution approchée, c'est-à-dire qui vérifie : 	$$x^2 - y \pmod{p} \ \le m.$$
	Par exemple si $m=5$, alors on peut avoir (modulo $p$) : $x^2 - y = 0$, $x^2 - y = 1$,
	$x^2-y = 2$,\ldots{} ou $x^2-y=5$. 
	
	Programme une fonction \ci{racine_approchee(y,marge)} qui trouve une solution approchée à notre problème $x^2 = y \pmod{p}$.
	
	Combien de temps faut-il pour trouver un solution au problème approché lorsque $p = 15\,486\,869$,  $y = 8\,371\,779$ et $m=20$ ?
	Choisis un nombre premier $p$ assez grand et trouve une marge d'erreur $m$ de sorte que trouver une solution au problème approché nécessite environ entre 30 et 60 secondes de calculs (pour un $y$ quelconque).

\end{enumerate}
  

Voici des exemples de nombres premiers que tu peux utiliser pour tes essais :
$$101 \quad
1097 \quad
10651 \quad
100109 \quad
1300573 \quad
15486869$$
$$179426321 \quad  
2276856017 \quad 
32416187567$$
\end{activite}



%%%%%%%%%%%%%%%%%%%%%%%%%%%%%%%%%%%%%%%%%%%%%%%%%%%%%%%%%%%%%%%%
%%%%%%%%%%%%%%%%%%%%%%%%%%%%%%%%%%%%%%%%%%%%%%%%%%%%%%%%%%%%%%%%

\begin{cours}[Chronomètre]

\index{temps de calcul}
\index{module!time@\ci{time}}
\index{time@\ci{time}}

Le module \ci{time} permet de chronométrer le temps d'exécution mais aussi de savoir la date et l'heure (voir aussi le module \ci{timeit} expliqué dans la fiche \og{}Arithmétique -- Tant que -- I\fg{}).
Voici un petit script pour mesurer le temps de calcul d'une instruction. 

\begin{lstlisting}
import time

debut_chrono = time.time()

time.sleep(2)

fin_chrono = time.time()

total_chrono = fin_chrono - debut_chrono
print("Temps d'exécution (en secondes) :", total_chrono)
\end{lstlisting}


\textbf{Explications.}
\begin{itemize}
  \item Le module s'appelle \ci{time}.
  \item La fonction qui renvoie le temps s'appelle aussi \ci{time()}, on l'appelle donc par \ci{time.time()} !
  \item La fonction \ci{time()} renvoie un nombre qui ne nous intéresse pas. Ce qui nous intéresse c'est l'écart entre deux valeurs de cette fonction. 
  \item \ci{debut_chrono = time.time()} c'est comme lancer un chronomètre.
  \item \ci{fin_chrono = time.time()} c'est comme l'arrêter.
  \item \ci{total_chrono = fin_chrono - debut_chrono} est la durée totale qu'a pris notre script en secondes.
  \item La fonction \ci{time.sleep(duree)} fait une pause d'une certaine durée de secondes dans le programme.
\end{itemize}
\end{cours}


%%%%%%%%%%%%%%%%%%%%%%%%%%%%%%%%%%%%%%%%%%%%%%%%%%%%%%%%%%%%%%%%
%%%%%%%%%%%%%%%%%%%%%%%%%%%%%%%%%%%%%%%%%%%%%%%%%%%%%%%%%%%%%%%%

\begin{cours}[\emph{Bitcoin} et \emph{blockchain}]

La monnaie \emph{bitcoin} est une monnaie dématérialisée. Les transactions sont enregistrées dans un grand livre de compte appelé \emph{blockchain}.\index{blockchain@\emph{blockchain}} Nous allons construire un modèle (très simplifié) d'un tel livre de compte.


Imaginons un groupe d'amis qui souhaitent partager les dépenses du groupe de façon la plus simple possible. Au départ tout le monde dispose de 1000 \emph{bitcoins} et on note au fur et à mesure les dépenses et les recettes de chacun. 
%À la fin des vacances, ils  régulariseront la situation.

On note sur le livre de compte la liste des dépenses/recettes, par exemple :
\begin{itemize}
  \item \og{}Amir a dépensé 100 \emph{bitcoins}\fg{}
  \item \og{}Barbara a reçu 45 \emph{bitcoins}\fg{}
  \item etc.
\end{itemize}
Il suffit de parcourir tout le livre pour savoir combien chacun a reçu ou dépensé depuis le début.

\bigskip

Pour éviter que quelqu'un ne vienne truquer le livre de compte, après chaque transaction on ajoute dans le livre une certification construite à partir d'une preuve de travail.
Voici ce que l'on écrit dans le livre :
\begin{enumerate}
  \item On commence par une preuve de travail quelconque. Pour nous ce sera 
  \ci{[0,0,0,0,0,0]}.
  \item On écrit la première transaction (par exemple \ci{"Amir -100"}).
  \item On calcule et on écrit dans le livre une preuve de travail, qui va servir de certificat. C'est une liste
  (par exemple \ci{[56,42,10,98,2,34]}) obtenue après beaucoup de calculs prenant en compte la transaction précédente et la précédente preuve de travail.
  \item À chaque nouvelle transaction (par exemple \ci{"Barbara +45"}), quelqu'un calcule une preuve de travail pour la dernière transaction associée à la précédente preuve. On écrit la transaction, puis la preuve de travail.
\end{enumerate}

\myfigure{0.5}{
\tikzinput{fig-bitcoin-4}
}  


La preuve de travail que l'on calcule dépend de la transaction précédente mais aussi de la preuve de travail précédente, qui elle même dépend des données antérieures... Ainsi, chaque nouvelle preuve de travail dépend en fait de tout ce qui a été écrit depuis le début (même si le calcul ne porte explicitement que sur les deux dernières entrées).

Quelqu'un qui voudrait truquer une transaction passée devrait recalculer toutes les preuves de travail qui viennent après. Ceci n'est pas possible pour une personne seule : il y a plusieurs preuves de travail à calculer et chaque preuve demande beaucoup de calculs.
\end{cours}


%%%%%%%%%%%%%%%%%%%%%%%%%%%%%%%%%%%%%%%%%%%%%%%%%%%%%%%%%%%%%%%%
% Activité 2
%%%%%%%%%%%%%%%%%%%%%%%%%%%%%%%%%%%%%%%%%%%%%%%%%%%%%%%%%%%%%%%%

\begin{activite}[Outils pour les listes]

\objectifs{Objectifs : construire des fonctions utiles pour manipuler les listes d'entiers dans les activités suivantes.}

Nos listes sont des listes d'entiers compris entre $0$ et $99$.

\begin{enumerate}
  \item Programme une fonction \ci{addition(liste1,liste2)} qui additionne terme à terme et modulo $100$, les éléments de deux listes de même longueur. 
  Par exemple \ci{addition([1,2,3,4,5,6],[1,1,1,98,98,98])} renvoie \ci{[2,3,4,2,3,4]}.
  
  \item On va chercher des listes qui commencent par des zéros (ou bien des zéros puis des nombres assez petits).  Une liste \ci{liste} est plus petite que la liste \ci{liste_max} si chaque élément de \ci{liste}  est inférieur ou égal à chaque élément de même rang de \ci{liste_max}.
  
  Par exemple la liste \ci{[0,0,1,2,3,4]} est plus petite que la liste \ci{[0,0,5]}.
  Ce n'est pas le cas de la liste \ci{[0,10,0,1,1]}.
  Autre exemple : être plus petit que la liste \ci{[0,0,0]} cela signifie commencer par trois zéros. Être plus petit que la liste \ci{[0,0,1]} signifie commencer par \ci{[0,0,0]} ou \ci{[0,0,1]}.
  
  Programme une fonction \ci{est_plus_petit(liste,liste_max)} qui renvoie \og{}vrai\fg{} lorsque \ci{liste} est plus petite que \ci{liste_max}.

  \item On aura besoin de transformer une phrase en une liste de nombres. De plus, on va découper nos listes en blocs de taille $N$ (avec $N=6$), on rajoute donc des zéros en début de liste afin qu'elle soit de longueur un multiple de $N$.
  
  Écris une fonction \ci{phrase_vers_liste(phrase)} qui convertit une chaîne de caractères en une liste d'entiers entre $0$ et $99$ et si besoin rajoute des zéros devant afin que la liste ait la bonne taille. 
  
  La formule à utiliser pour convertir un caractère en un entier strictement inférieur à $100$ est :  
  \mycenterline{\ci{ord(c) \% 100}}  
  
  Par exemple : si \ci{phrase = "Vive moi !"} alors la fonction renvoie :   
\mycenterline{\ci{[0, 0, 86, 5, 18, 1, 32, 9, 11, 5, 32, 33]}}

Le caractère \ci{"i"} à pour code ASCII/unicode $105$ donc, modulo $100$, le nombre est $5$. Note que la fonction ajoute deux $0$ en début de liste afin d'avoir une longueur qui est un multiple de $N=6$.
  
\end{enumerate}   
     
\end{activite}



%%%%%%%%%%%%%%%%%%%%%%%%%%%%%%%%%%%%%%%%%%%%%%%%%%%%%%%%%%%%%%%%
% Activité 3
%%%%%%%%%%%%%%%%%%%%%%%%%%%%%%%%%%%%%%%%%%%%%%%%%%%%%%%%%%%%%%%%

\begin{activite}[Fonction de hachage]

\objectifs{Objectifs : créer une fonction de hachage. À partir d'un long message nous calculons une courte empreinte. Il est difficile de trouver deux messages différents ayant la même empreinte.}

\index{fonction!de hachage}
\index{hash@\emph{hash}}

Dans cette activité, notre message est une liste d'entiers (entre $0$ et $99$) de longueur un multiple quelconque de $N=6$, nous le transformons en une liste de longueur $N=6$ : son empreinte (ou \emph{hash}). Voici des exemples de ce que va faire notre fonction de hachage :
\begin{itemize}	
  \item la liste \ci{[1, 2, 3, 4, 5, 6, 1, 2, 3, 4, 5, 6]} a pour empreinte :  
  \mycenterline{\ci{[10, 0, 58, 28, 0, 90]}}
  
  \item la liste \ci{[1, 1, 3, 4, 5, 6, 1, 2, 3, 4, 5, 6]} a pour empreinte :  
  \mycenterline{\ci{[25, 14, 29, 1, 19, 6]}}
\end{itemize}  
  
 L'idée est de mélanger les nombres par bloc de $N=6$ entiers, puis de combiner ce bloc au suivant et de recommencer, jusqu'à obtenir un seul bloc.
\begin{enumerate}


  \item \textbf{Un tour.} 
  
  Pour un bloc $[b_0,b_1,b_2,b_3,b_4,b_5]$ de taille $N=6$, \emph{faire un tour} consiste à faire les opérations suivantes :
  \begin{enumerate}
    \item On additionne certains entiers : 
    $$[b_0',b_1',b_2',b_3',b_4',b_5'] = [b_0,b_1+b_0,b_2,b_3+b_2,b_4,b_5+b_4]$$
    
    \item On multiplie ces entiers par des nombres premiers (dans l'ordre $7,11,13,17,19,23$) et on rajoute $1$ :
    $$[b_0'',b_1'',b_2'',b_3'',b_4'',b_5''] = [7 \times b_0'+1,11\times b_1'+1,13\times b_2'+1,17 \times b_3'+1,19 \times b_4'+1,23 \times b_5'+1]$$
    
    \item On effectue une permutation circulaire (le dernier passe devant) :
    $$[b_0''',b_1''',b_2''',b_3''',b_4''',b_5'''] = [b_5'',b_0'',b_1'',b_2'',b_3'',b_4'']$$
    
    \item On réduit chaque entier modulo $100$ afin d'obtenir des entiers entre $0$ et $99$.
  \end{enumerate}


\myfigure{0.7}{
\tikzinput{fig-bitcoin-1}
}  

  
  Partant du bloc $[0, 1, 2, 3, 4, 5]$, on a donc successivement :
   \begin{enumerate}
    \item additions : $[0, 1, 2, 5, 4, 9]$
    
    \item multiplications : $[7\times 0 + 1, 11\times 1+1, 13\times 2+1, 17 \times 5+1,19 \times  4+1, 23 \times 9+1] = [1,12,27,86,77,208]$ 
    
    \item permutation : $[208,1,12,27,86,77]$
    
    \item réduction modulo $100$ : $[8,1,12,27,86,77]$
  \end{enumerate}
  
  Programme un telle fonction \ci{un_tour(bloc)} qui renvoie la transformation du bloc après ces opérations. Vérifie que le bloc $[1, 1, 2, 3, 4, 5]$ est transformé en $[8, 8, 23, 27, 86, 77]$.
 
    \item \textbf{Dix tours.} 
    
    Pour bien mélanger chaque bloc, programme une fonction  \ci{dix_tours(bloc)} qui itère dix fois les opérations précédentes.
    Après $10$ tours :
    \begin{itemize}
      \item le bloc $[0, 1, 2, 3, 4, 5]$ devient $[98, 95, 86, 55, 66, 75]$,
      \item le bloc $[1, 1, 2, 3, 4, 5]$ devient $[18, 74, 4, 42, 77, 42]$.
    \end{itemize}
 Deux blocs proches sont transformés en deux blocs très différents ! 
   
   \item \textbf{Hachage d'une liste.} 
   
   Partant d'une liste de longueur un multiple de $N=6$, on la découpe en blocs de longueur $6$ et on calcule l'empreinte de cette liste selon l'algorithme suivant :
   \begin{itemize}
     \item On extrait le premier bloc de la liste, on effectue $10$ tours de mélange.
     \item On ajoute terme à terme (et modulo $100$), le résultat de ce mélange au second bloc.
     \item On recommence en partant du nouveau second bloc.
     \item Lorsqu'il ne reste plus qu'un bloc, on effectue $10$ tours de mélange, le résultat est l'empreinte de la liste.
   \end{itemize}
 
 
Voici le schéma d'une situation avec trois blocs : dans un premier temps il y a trois blocs (A,B,C) ; dans un second temps il ne reste plus que deux blocs (B' et C) ; à la fin il ne reste qu'un bloc (C'') : c'est l'empreinte !
\myfigure{0.5}{
\tikzinput{fig-bitcoin-2}
}  
  
  
  Exemple avec la liste $[0,1,2,3,4,5,1,1,1,1,1,1,10,10,10,10,10,10]$.
  \begin{itemize}
    \item Le premier bloc est $[0,1,2,3,4,5]$, son mélange à $10$ tours est 
    $[98, 95, 86, 55, 66, 75]$. 
    \item On ajoute ce mélange au second bloc $[1,1,1,1,1,1]$ (voir la fonction \ci{addition()} de l'activité 2).
    \item La liste restante est maintenant $[99,96,87,56,67,76,10,10,10,10,10,10]$.
    \item On recommence. Le nouveau premier bloc est $[99,96,87,56,67,76]$, son mélange à $10$ tours vaut $[60, 82, 12, 94, 6, 80]$, on l'ajoute au dernier bloc $[10,10,10,10,10,10$] pour obtenir (modulo $100$) $[70,92,22,4,16,90]$.
    \item On effectue un dernier mélange à $10$ tours pour obtenir l'empreinte : $[77, 91, 5, 91, 89, 99]$.
   \end{itemize}
   
   Programme une fonction \ci{hachage(liste)} qui renvoie l'empreinte d'une liste.
     Teste-la sur les exemples donnés au début de l'activité.
\end{enumerate}   
     
\end{activite}



%%%%%%%%%%%%%%%%%%%%%%%%%%%%%%%%%%%%%%%%%%%%%%%%%%%%%%%%%%%%%%%%
% Activité 4
%%%%%%%%%%%%%%%%%%%%%%%%%%%%%%%%%%%%%%%%%%%%%%%%%%%%%%%%%%%%%%%%

\begin{activite}[Preuve de travail - Minage]

\objectifs{Objectifs : construire un mécanisme de preuve de travail à l'aide de notre fonction de hachage.}

\index{minage}

On va construire un problème compliqué à résoudre, pour lequel, si quelqu'un nous donne la solution, alors il est facile de vérifier qu'elle convient.

\medskip

\textbf{Problème à résoudre.} On nous donne une liste, il s'agit de trouver un bloc tel que, lorsque qu'on le rajoute à la liste, cela produit un hachage commençant par des zéros.
Plus précisément étant donné une liste  \ci{liste} et un objectif maximal \ci{Max}, il s'agit de trouver un bloc \ci{preuve} qui, concaténé à la liste puis haché, est plus petit que la liste \ci{Max}, c'est-à-dire : 
\mycenterline{\ci{hachage(liste + preuve)} \  plus petit que \  \ci{Max}}

La liste est de longueur quelconque (un multiple de $N=6$), la preuve est un bloc de longueur $N$, l'objectif est de trouver une liste commençant par des $0$ (voir l'activité 2).

\medskip

Par exemple : soit la \ci{liste = [0,1,2,3,4,5]} et \ci{Max = [0,0,7]}. Quel bloc \ci{preuve} puis-je concaténer à \ci{liste} pour résoudre mon problème ?
\begin{itemize}
  \item \ci{preuve = [12,3,24,72,47,77]} convient car concaténé à notre liste cela donne \ci{[0,1,2,3,4,5,12,3,24,72,47,77]} et le hachage de toute cette liste vaut
  \ci{[0,0,5,47,44,71]} qui commence par \ci{[0,0,5]} plus petit que l'objectif.

  \item \ci{preuve = [0,0,2,0,61,2]} convient aussi car après concaténation on a 
  \ci{[0,1,2,3,4,5,0,0,2,0,61,2]} dont le hachage donne \ci{[0,0,3,12,58,92]}.
  
  \item \ci{[97,49,93,87,89,47]} ne convient pas, car après concaténation puis hachage on obtient 
  \ci{[0,0,8,28,6,60]} qui est plus grand que l'objectif voulu.
\end{itemize}

\bigskip

\begin{enumerate}
  \item \textbf{Vérification (facile).} 
  
  Programme une fonction \ci{verification_preuve_de_travail(liste,preuve)} qui renvoie vrai si la solution \ci{preuve} proposée convient pour \ci{liste}. Utilise la fonction \ci{est_plus_petit()} de l'activité 2.
  
  \item \textbf{Recherche de solution (difficile).}
  
  Programme une fonction \ci{preuve_de_travail(liste)} qui cherche un bloc \ci{preuve} solution à notre problème pour la liste donnée. 
  
  
  \emph{Indications.}
  
  \begin{itemize}
    \item La méthode la plus simple est de prendre un bloc \ci{preuve} de nombres au hasard et de recommencer jusqu'à trouver une solution.
    
    \item Tu peux aussi tester systématiquement tous les blocs en commençant avec \ci{[0,0,0,0,0,0]}, puis \ci{[0,0,0,0,0,1]}\ldots{} et t'arrêter au premier qui convient.
    
    \item Tu ajustes la difficulté du problème en changeant l'objectif : facile avec \ci{Max = [0,0,50]}, moyen avec \ci{Max = [0,0,5]}, difficile avec \ci{Max = [0,0,0]}, trop difficile avec \ci{Max = [0,0,0,0]}.
    
    \item Comme il existe plusieurs solutions, tu n'obtiens pas nécessairement la même solution à chaque recherche. 
    
   \end{itemize}
   
   \item \textbf{Temps de calcul.}
   
   Compare le temps de calcul d'une simple vérification par rapport au temps de recherche d'une solution. Choisis l'objectif \ci{Max} de sorte que la recherche d'une preuve de travail nécessite environ entre 30 et 60 secondes de calculs.
  
\end{enumerate}  

Pour le \emph{bitcoin}, ceux qui calculent des preuves de travail sont appelés les \emph{mineurs}. Le premier qui trouve une preuve gagne une récompense. La difficulté du problème est ajustée de sorte 
que le temps de calcul mis par le gagnant (parmi l'ensemble de tous les mineurs) pour trouver une solution, soit d'environ $10$ minutes.

\end{activite}


%%%%%%%%%%%%%%%%%%%%%%%%%%%%%%%%%%%%%%%%%%%%%%%%%%%%%%%%%%%%%%%%
% Activité 4
%%%%%%%%%%%%%%%%%%%%%%%%%%%%%%%%%%%%%%%%%%%%%%%%%%%%%%%%%%%%%%%%

\begin{activite}[Tes \emph{bitcoins}]

\objectifs{Objectifs : créer un livre de compte (appelé \emph{blockchain} pour le \emph{bitcoin}) qui enregistre toutes les transactions, ce registre est public et certifié. Il est pratiquement impossible d'y falsifier une transaction déjà inscrite.}


\myfigure{0.5}{
\tikzinput{fig-bitcoin-3}
}  

\begin{enumerate}
  \item \textbf{Initialisation et ajout d'une transaction.}
  
  \begin{enumerate}
    \item Initialise une variable globale \ci{Livre} qui est une liste et contient au départ une preuve nulle : \ci{Livre = [ [0,0,0,0,0,0] ]} .
  
    \item Une \emph{transaction} est une chaîne de caractères comprenant un nom et la somme à ajouter (ou à retrancher) à son compte. Par exemple \ci{"Abel +25"} ou \ci{"Barbara -45"}.
    
    Programme une fonction \ci{ajout_transaction(transaction)} qui ajoute la chaîne de caractères \ci{transaction} à la liste \ci{Livre}. 
Par exemple après l'initialisation \ci{ajout_transaction("Camille +100")},  
    \ci{Livre} vaut \ci{[ [0,0,0,0,0,0], "Camille +100" ]}.
    Attention, pour pouvoir modifier \ci{Livre} il faut commencer la fonction par : \ci{global Livre}.
    
  \end{enumerate} 
  
   \item Dès qu'une transaction est ajoutée, il faut calculer et ajouter au livre de compte une preuve de travail. Programme une fonction \ci{minage()}, sans paramètre, qui ajoute une preuve de travail au livre.   
   
Voici comment faire :
  \begin{itemize}
    \item On prend la dernière transaction \ci{transaction}, on la transforme en une liste d'entiers par la fonction \ci{phrase_vers_liste()} de l'activité 2.
    \item On prend la preuve de travail \ci{prec_preuve} située juste avant cette transaction.
    \item On forme la liste \ci{liste} composée d'abord des éléments de \ci{prec_preuve}, puis des éléments de la liste d'entiers obtenue en convertissant la chaîne \ci{transaction}.
    \item On calcule une preuve de travail de cette liste.
    \item  On ajoute cette preuve au livre de compte.
  \end{itemize}
  Par exemple si le livre se termine par :
  \mycenterline{\ci{[3,1,4,1,5,9], "Abel +35"}}
  
  alors après calcul de la preuve de travail le livre se termine par exemple par :  
  \mycenterline{\ci{[3,1,4,1,5,9], "Abel +35", [32,17,37,73,52,90]}}  
  
     On rappelle que la preuve de travail n'est pas unique et qu'en plus elle dépend 
   de l'objectif \ci{Max}.
   
   \item Une seule personne à la fois ajoute une preuve de travail. Par contre tout le monde a la possibilité de vérifier que la preuve proposée est correcte (et devrait le faire). 
	Écris une fonction \ci{verification_livre()}, sans paramètre, qui vérifie que la dernière preuve ajoutée au \ci{Livre} est valide.
	
	\item Écris un livre de compte qui correspond aux données suivantes :
	\begin{itemize}
	  \item On prend \ci{Max  = [0,0,5]} et au départ \ci{Livre = [ [0,0,0,0,0,0] ]}.
	  \item \ci{"Alfred -100"} (Alfred doit 100 \emph{bitcoins}).
	  \item Barnabé en reçoit 150.
	  \item Chloé gagne 35 \emph{bitcoins}.
	\end{itemize}

\end{enumerate} 

Conclusion : imaginons que Alfred veuille tricher, il veut changer le livre de compte afin de recevoir 100 \emph{bitcoins} au lieu d'en devoir 100. Il doit donc changer la transaction le concernant en \ci{"Alfred +100"} mais il doit alors recalculer une nouvelle preuve de travail ce qui est compliqué, surtout il doit aussi recalculer la preuve de la transaction de Barnabé et aussi celle de la transaction de Chloé ! 

Quelqu'un qui veut modifier une transaction doit modifier toutes les preuves de travail suivantes. Si chaque preuve demande suffisamment de temps de calcul ceci est impossible. Pour le \emph{bitcoin} chaque preuve demande énormément de calculs (trop pour une personne seule) et une nouvelle preuve est à calculer toutes les 10 minutes. Il est donc impossible pour une personne de modifier une transaction passée.

\end{activite}


L'autre aspect du \emph{bitcoin} que nous n'avons pas abordé, c'est de s'assurer de l'identité de chaque personne impliquée, afin que personne ne puisse récupérer l'argent d'un autre. Ceci est rendu possible grâce à la cryptographie à clé privée/clé publique (système RSA). Chaque compte est identifié par une clé publique (deux très grands entiers), ce qui garantit l'anonymat. Mais surtout, seul celui qui possède la clé privée du compte (un grand entier) peut accéder à ses \emph{bitcoins}.

\end{document}
