\documentclass[12pt,class=report,crop=false]{standalone}
\usepackage[screen]{../python}


\pagestyle{empty}

\begin{document}


%====================================================================
\chapitre{Bitcoin}
%====================================================================

\section*{Bitcoin}

\begin{itemize}
   \item La monnaie \emph{bitcoin} est une monnaie dématérialisée. 

   \item Les transactions sont enregistrées dans un grand livre de compte appelé \emph{blockchain}. 


   \item Imaginons un groupe d'amis qui souhaitent partager les dépenses du groupe de façon la plus simple possible.

   \item Au départ tout le monde dispose de 1000 \emph{bitcoins} et on note au fur et à mesure les dépenses et les recettes de chacun.
 
   \item On note sur le livre de compte la liste des dépenses/recettes, par exemple :
\begin{itemize}
  \item \og{}Amir a dépensé 100 \emph{bitcoins}\fg{}
  \item \og{}Barbara a reçu 45 \emph{bitcoins}\fg{}
  \item etc.
\end{itemize}

   \item Il suffit de parcourir tout le livre pour savoir combien chacun a reçu ou dépensé depuis le début.
\end{itemize}

\newpage

\section*{Blockchain}

Pour éviter que quelqu'un ne vienne truquer le livre de compte, après chaque transaction on ajoute dans le livre une certification construite à partir d'une preuve de travail.
\begin{enumerate}
  \item On commence par une preuve de travail quelconque. Pour nous ce sera 
  \ci{[0,0,0,0,0,0]}.
  \item On écrit la première transaction (par exemple \ci{"Amir -100"}).
  \item On calcule et on écrit dans le livre une preuve de travail, qui va servir de certificat. C'est une liste
  (par exemple \ci{[56,42,10,98,2,34]}) obtenue après beaucoup de calculs prenant en compte la transaction précédente et la précédente preuve de travail.
  \item À chaque nouvelle transaction (par exemple \ci{"Barbara +45"}), quelqu'un calcule une preuve de travail pour la dernière transaction associée à la précédente preuve. On écrit la transaction, puis la preuve de travail.
\end{enumerate}

\myfigure{0.5}{
\tikzinput{fig-bitcoin-4}
}  

\newpage


\section*{Preuve de travail}



\begin{itemize}
   \item Une preuve de travail est la résolution d'un problème difficile, mais où il est facile de vérifier que la solution obtenue est correcte.

  \item Comme les sudokus par exemple : il suffit de dix secondes pour vérifier qu'une grille est remplie correctement, par contre il a fallu plus de dix minutes pour le résoudre.
\end{itemize}

\myfigure{0.5}{
\tikzinput{fig-bitcoin-4}
}  

\newpage

\section*{Fonction de hachage}


\begin{itemize}
  \item À partir d'un long message nous calculons une courte empreinte. 
  \item Il est difficile de trouver deux messages différents ayant la même empreinte.
  \item Ici notre message est une liste d'entiers (entre $0$ et $99$) de longueur un multiple quelconque de $N=6$.
  \item Son empreinte (ou \emph{hash}) sera une liste de longueur $N=6$.
  
  \item  Exemple : \ci{[1, 2, 3, 4, 5, 6, 1, 2, 3, 4, 5, 6]} a pour empreinte :\\
  \centerline{\ci{[10, 0, 58, 28, 0, 90]}}
  \item Exemple : \ci{[1, 1, 3, 4, 5, 6, 1, 2, 3, 4, 5, 6]} a pour empreinte :\\
  \centerline{\ci{[25, 14, 29, 1, 19, 6]}}
\end{itemize}  
  
 L'idée est de mélanger les nombres par bloc de $N=6$ entiers, puis de combiner ce bloc au suivant et de recommencer, jusqu'à obtenir un seul bloc.

\newpage

\section*{Un tour} 
  
  Pour un bloc $[b_0,b_1,b_2,b_3,b_4,b_5]$ de taille $N=6$, \emph{faire un tour} consiste à faire les opérations suivantes :
  \begin{enumerate}
    \item On additionne certains entiers : 
    $$[b_0',b_1',b_2',b_3',b_4',b_5'] = [b_0,b_1+b_0,b_2,b_3+b_2,b_4,b_5+b_4]$$
    
    \item On multiplie ces entiers par des nombres premiers (dans l'ordre $7,11,13,17,19,23$) et on rajoute $1$ :
    $$[b_0'',b_1'',b_2'',b_3'',b_4'',b_5''] = [7 \times b_0'+1,11\times b_1'+1,13\times b_2'+1,17 \times b_3'+1,19 \times b_4'+1,23 \times b_5'+1]$$
    
    \item On effectue une permutation circulaire (le dernier passe devant) :
    $$[b_0''',b_1''',b_2''',b_3''',b_4''',b_5'''] = [b_5'',b_0'',b_1'',b_2'',b_3'',b_4'']$$
    
    \item On réduit chaque entier modulo $100$ afin d'obtenir des entiers entre $0$ et $99$.
  \end{enumerate}


\myfigure{0.7}{
\tikzinput{fig-bitcoin-1}
}  


\newpage
\begin{minipage}{0.6\textwidth}  
  Partant du bloc $[0, 1, 2, 3, 4, 5]$, on a donc successivement :
   \begin{enumerate}
    \item additions : $[0, 1, 2, 5, 4, 9]$
    
    \item multiplications : $[7\times 0 + 1, 11\times 1+1, 13\times 2+1, 17 \times 5+1,19 \times  4+1, 23 \times 9+1] = [1,12,27,86,77,208]$ 
    
    \item permutation : $[208,1,12,27,86,77]$
    
    \item réduction modulo $100$ : $[8,1,12,27,86,77]$
  \end{enumerate}
  
   Vérifie que le bloc $[1, 1, 2, 3, 4, 5]$ est transformé en $[8, 8, 23, 27, 86, 77]$.
\end{minipage}\quad
\begin{minipage}{0.3\textwidth}
 \myfigure{0.7}{
\tikzinput{fig-bitcoin-1}
}  
\end{minipage} 
 
 
\newpage

 
\section*{Dix tours} 
    
    Pour bien mélanger chaque bloc, itère dix fois les opérations précédentes.
    Après $10$ tours :
    \begin{itemize}
      \item le bloc $[0, 1, 2, 3, 4, 5]$ devient $[98, 95, 86, 55, 66, 75]$,
      \item le bloc $[1, 1, 2, 3, 4, 5]$ devient $[18, 74, 4, 42, 77, 42]$.
    \end{itemize}
 Deux blocs proches sont transformés en deux blocs très différents ! 
   
\newpage   
   
\section*{Hachage d'une liste} 
   
   Partant d'une liste de longueur un multiple de $N=6$, on la découpe en blocs de longueur $6$ et on calcule l'empreinte de cette liste selon l'algorithme suivant :
   \begin{itemize}
     \item On extrait le premier bloc de la liste, on effectue $10$ tours de mélange.
     \item On ajoute terme à terme (et modulo $100$), le résultat de ce mélange au second bloc.
     \item On recommence en partant du nouveau second bloc.
     \item Lorsqu'il ne reste plus qu'un bloc, on effectue $10$ tours de mélange, le résultat est l'empreinte de la liste.
   \end{itemize}
 
 
Voici le schéma d'une situation avec trois blocs : dans un premier temps il y a trois blocs (A,B,C) ; dans un second temps il ne reste plus que deux bloc (B' et C) ; à la fin il ne reste qu'un bloc (C'') : c'est l'empreinte !
\myfigure{0.5}{
\tikzinput{fig-bitcoin-2}
}  
  
\newpage
  
  Exemple avec la liste $[0,1,2,3,4,5,1,1,1,1,1,1,10,10,10,10,10,10]$.
  \begin{itemize}
    \item Le premier bloc est $[0,1,2,3,4,5]$, son mélange à $10$ tours est 
    $[98, 95, 86, 55, 66, 75]$. 
    \item On ajoute ce mélange au second bloc $[1,1,1,1,1,1]$.
    \item La liste restante est maintenant $[99,96,87,56,67,76,10,10,10,10,10,10]$.
    \item On recommence. Le nouveau premier bloc est $[99,96,87,56,67,76]$, son mélange à $10$ tours vaut $[60, 82, 12, 94, 6, 80]$, on l'ajoute au dernier bloc $[10,10,10,10,10,10$] pour obtenir (modulo $100$) $[70,92,22,4,16,90]$.
    \item On effectue un dernier mélange à $10$ tours pour obtenir l'empreinte : $[77, 91, 5, 91, 89, 99]$.
   \end{itemize}
   
   
\myfigure{0.4}{
\tikzinput{fig-bitcoin-2}
}     

\newpage

%%%%%%%%%%%%%%%%%%%%%%%%%%%%%%%%%%%%%%%%%%%%%%%%%%%%%%%%%%%%%%%%
% Activité 4
%%%%%%%%%%%%%%%%%%%%%%%%%%%%%%%%%%%%%%%%%%%%%%%%%%%%%%%%%%%%%%%%

\section*{Preuve de travail - Minage}


On va construire un problème compliqué à résoudre, pour lequel, si quelqu'un nous donne la solution, alors il est facile de vérifier qu'elle convient.

\bigskip

\textbf{Problème à résoudre.} On nous donne une liste, il s'agit de trouver un bloc tel que, lorsque qu'on le rajoute à la liste, cela produit un hachage commençant par des zéros.
Plus précisément étant donné une liste  \ci{liste} et un objectif maximal \ci{Max}, il s'agit de trouver un bloc \ci{preuve} qui, concaténé à la liste puis haché, est plus petit que la liste \ci{Max}, c'est-à-dire : \\
\centerline{\ci{hachage(liste + preuve)} \  plus petit que \  \ci{Max}}


La liste est de longueur quelconque (un multiple de $N=6$), la preuve est un bloc de longueur $N$, l'objectif est de trouver une liste commençant par des $0$.

\newpage

\textbf{Exemple.}

Soit la \ci{liste = [0,1,2,3,4,5]} et \ci{Max = [0,0,7]}. Quel bloc \ci{preuve} puis-je concaténer à \ci{liste} pour résoudre mon problème ?

\bigskip

\begin{itemize}
  \item 
  \begin{itemize}
  \item   \ci{preuve = [12, 3, 24, 72, 47, 77]} convient 
  \item car concaténé à notre liste cela donne \ci{[0,1,2,3,4,5,12, 3, 24, 72, 47, 77]} 
  \item et le hachage de toute cette liste vaut
  \ci{[0, 0, 5, 47, 44, 71]} 
  \item qui commence par \ci{[0,0,5]} plus petit que l'objectif.
  \end{itemize}

\bigskip

  \item 
    \begin{itemize}
  \item  
  \ci{preuve = [0, 0, 2, 0, 61, 2]} convient aussi 
  \item car après concaténation on a 
  \ci{[0,1,2,3,4,5,0, 0, 2, 0, 61, 2]} 
  \item dont le hachage donne \ci{[0, 0, 3, 12, 58, 92]}.
    \end{itemize}
    
  \bigskip
    
  \item \ci{[97, 49, 93, 87, 89, 47]} ne convient pas, car après concaténation puis hachage on obtient \ci{[0, 0, 8, 28, 6, 60]} qui est plus grand que l'objectif voulu.
\end{itemize}


\newpage

Variable globale \ci{Max = [0,0,7]}

  \bigskip
  
\textbf{Vérification (facile).} 
  
  Programme une fonction \ci{verification_preuve_de_travail(liste,preuve)} qui renvoie vrai si la solution \ci{preuve} proposée convient pour \ci{liste}. 
   
  \bigskip
  
\textbf{Recherche de solution (difficile).}
  
  Programme une fonction \ci{preuve_de_travail(liste)} qui cherche un bloc \ci{preuve} solution à notre problème pour la liste donnée. 
  
  \bigskip
  
  \emph{Indications.}
  
  \begin{itemize}
    \item La méthode la plus simple est de prendre un bloc \ci{preuve} de nombres au hasard et de recommencer jusqu'à trouver une solution.
    
    \item Tu peux aussi tester systématiquement tous les blocs en commençant avec \ci{[0,0,0,0,0,0]}, puis \ci{[0,0,0,0,0,1]}\ldots{} et t'arrêter au premier qui convient.
    
    \item Tu ajustes la difficulté du problème en changeant l'objectif : facile avec \ci{Max = [0,0,50]}, moyen avec \ci{Max = [0,0,5]}, difficile avec \ci{Max = [0,0,0]}, trop difficile avec \ci{Max = [0,0,0,0]}.
    
    \item Comme il existe plusieurs solutions, tu n'obtiens pas nécessairement la même solution à chaque recherche. 
    
   \end{itemize}
   
\newpage   

 \textbf{Temps de calcul.}
   
   \begin{itemize}
   \item  Compare le temps de calcul d'une simple vérification par rapport au temps de recherche d'une solution. 
   
   \item Choisis l'objectif \ci{Max} de sorte que la recherche d'une preuve de travail nécessite environ entre 30 et 60 secondes de calculs.
  
  \item Pour le \emph{bitcoin}, ceux qui calculent des preuves de travail sont appelés les \emph{mineurs}. 
  
  \item Le premier qui trouve une preuve gagne une récompense. 
  
  \item La difficulté du problème est ajustée de sorte 
que le temps de calcul mis par le gagnant (parmi l'ensemble de tous les mineurs) pour trouver une solution, soit d'environ $10$ minutes.

  \end{itemize}

\newpage

%%%%%%%%%%%%%%%%%%%%%%%%%%%%%%%%%%%%%%%%%%%%%%%%%%%%%%%%%%%%%%%%
% Activité 4
%%%%%%%%%%%%%%%%%%%%%%%%%%%%%%%%%%%%%%%%%%%%%%%%%%%%%%%%%%%%%%%%

\section*{Tes \emph{bitcoins}}

\myfigure{0.5}{
\tikzinput{fig-bitcoin-3}
}  

\newpage

\textbf{Initialisation et ajout d'une transaction.}
  
  \begin{enumerate}
    \item Variable globale \ci{Livre} initialisée à \ci{Livre = [ [0,0,0,0,0,0] ]} .
  
    \item Une \emph{transaction} est une chaîne de caractères comprenant un nom et la somme à ajouter (ou à retrancher) à son compte. Par exemple \ci{"Abel +25"} ou \ci{"Barbara -45"}.
    
    Programme une fonction \ci{ajout_transaction(transaction)} qui ajoute la chaîne de caractère \ci{transaction} à la liste \ci{Livre}.
    
     
Par exemple après l'initialisation 
\ci{ajout_transaction("Camille +100")},  
    \ci{Livre} vaut \ci{[ [0, 0, 0, 0, 0, 0], "Camille +100" ]}.
    
    
    Attention, pour pouvoir modifier \ci{Livre} il faut commencer la fonction par : \ci{global Livre}.
 
 \newpage

 \textbf{Phrase vers liste d'entiers.}  
  
  Écris une fonction \ci{phrase_vers_liste(phrase)} qui convertit une chaîne de caractères en liste d'entiers entre $0$ et $99$ et si besoin rajoute des zéros devant afin que la liste ait une longueur un multiple de $N=6$. 
  
    \bigskip
    
  La formule à utiliser pour convertir un caractère en un entier strictement inférieur à $100$ est :\\
  \centerline{\ci{ord(c) \% 100}} 
  
  \bigskip
   
  Exemple : \\
  \centerline{\ci{phrase = "Vive moi !"}}
  
  la fonction renvoie :\\  
\centerline{\ci{[0, 0, 86, 5, 18, 1, 32, 9, 11, 5, 32, 33]}}


 
\newpage

 \textbf{Calcul et ajout d'une preuve de travail.}  

Dès qu'une transaction est ajoutée, il faut calculer et ajouter au livre de compte une preuve de travail. Programme une fonction \ci{minage()}, sans paramètre, qui ajoute une preuve de travail au livre.   
   
Voici comment faire :
  \begin{itemize}
    \item On prend la dernière transaction \ci{transaction}, on la transforme en une liste d'entiers par la fonction \ci{phrase_vers_liste()}.
    \item On prend la preuve de travail \ci{prec_preuve} située juste avant cette transaction.
    \item On forme la liste \ci{liste} composée d'abord des éléments de \ci{prec_preuve}, puis des éléments de la liste d'entiers obtenue en convertissant la chaîne \ci{transaction}.
    \item On calcule une preuve de travail de cette liste.
    \item  On ajoute cette preuve au livre de compte.
  \end{itemize}
  Par exemple si le livre se termine par :\\
  \centerline{\ci{[3, 1, 4, 1, 5, 9], "Abel +35"}}
  alors après calcul de la preuve de travail le livre se termine par exemple par :\\
  \centerline{\ci{[3, 1, 4, 1, 5, 9], "Abel +35", [32, 17, 37, 73, 52, 90]}}  
     On rappelle que la preuve de travail n'est pas unique et qu'en plus elle dépend 
   de l'objectif \ci{Max}.
   
 \newpage

 \textbf{Vérification du livre de compte.}  
   
Une seule personne à la fois ajoute une preuve de travail. Par contre tout le monde a la possibilité de vérifier que la preuve proposée est correcte (et devrait le faire). 


	Écris une fonction \ci{verification_livre()}, sans paramètre, qui vérifie que la dernière preuve ajoutée au \ci{Livre} est valide.
	
	
	\bigskip
	

\textbf{Exemple.}

Écris un livre de compte qui correspond aux données suivantes :
	\begin{itemize}
	  \item On prend \ci{Max  = [0,0,5]} et au départ \ci{Livre = [ [0,0,0,0,0,0] ]}.
	  \item \ci{"Alfred -100"} (Alfred doit 100 \emph{bitcoins}).
	  \item Barnabé en reçoit 150.
	  \item Chloé gagne 35 \emph{bitcoins}.
	\end{itemize}

\end{enumerate} 



\end{document}
