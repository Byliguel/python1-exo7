\documentclass[11pt,class=report,crop=false]{standalone}
\usepackage[screen]{../python}

\pagestyle{empty}

\begin{document}


%====================================================================
\chapitre{Premiers pas}
%====================================================================


\section*{Boucle \og{}pour\fg{}}


\bigskip

La boucle \og{}pour\fg{} est la façon la plus simple de répéter des instructions.

\bigskip



\mybox{
\myfigure{0.7}{
  \tikzinput{fig-premiers_pas-boucle-pour}
} }

\newpage

\section*{\ci{range()}}


\bigskip

	\begin{itemize}
	  \item avec \ci{range(n)} on parcourt les entiers de $0$ à $n-1$.
	  Par exemple \ci{range(10)} correspond à la liste \ci{[0, 1, 2, 3, 4, 5, 6, 7, 8, 9]}. 
	  
	  \item  Attention ! la liste s'arrête bien à $n-1$ et pas à $n$. Ce qu'il faut retenir 
	  c'est que la liste contient bien $n$ éléments (car elle commence à $0$).
	  
	 \bigskip 
	  
	 \item Si tu veux afficher la liste des éléments parcourus, il faut utiliser la commande :\\
	  \centerline{\ci{list(range(10))}}

\bigskip
	
	\item Avec \ci{range(a,b)} on parcourt les éléments de $a$ à $b-1$.
	Par exemple  \ci{range(10,20)} correspond à la liste \ci{[10, 11, 12, 13, 14, 15, 16, 17, 18, 19]}.  

\bigskip
	
	\item Avec \ci{range(a,b,pas)} on parcourt les éléments $a$, $a+\text{pas}$, $a + 2\text{pas}$\ldots{} Par exemple \ci{range(10,20,2)} correspond à la liste \ci{[10, 12, 14, 16, 18]}.  

	\end{itemize}
\end{document}
