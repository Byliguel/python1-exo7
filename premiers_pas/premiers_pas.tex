\documentclass[11pt,class=report,crop=false]{standalone}
\usepackage[screen]{../python}

\begin{document}


%====================================================================
\chapitre{Premiers pas}
%====================================================================


\objectifs{Lance-toi dans la programmation ! Dans cette toute première activité, tu vas apprendre à manipuler des nombres, des variables et tu vas coder tes premières boucles avec \Python.}


\insertvideo{9M9yz8bFzuU}{Premiers pas - partie 1 - Lance-toi !}

\insertvideo{9h9nLLRRjls}{Premiers pas - partie 2 - Variable}

\insertvideo{WLTNpIdHIls}{Premiers pas - partie 3 - Fonctions}

\insertvideo{zxf7QRfZato}{Premiers pas - partie 4 - Boucle pour}

\insertvideo{IS117_uXZKE}{Installer Python}

\insertvideo{QtUxL52VPtA}{Démarrer Python et utiliser IDLE}


%%%%%%%%%%%%%%%%%%%%%%%%%%%%%%%%%%%%%%%%%%%%%%%%%%%%%%%%%%%%%%%%
%%%%%%%%%%%%%%%%%%%%%%%%%%%%%%%%%%%%%%%%%%%%%%%%%%%%%%%%%%%%%%%%

\begin{cours}[Nombres avec \Python]

Vérifie dans la console que \Python{} fonctionne correctement, en tapant les commandes suivantes dans une console \Python{} :
\begin{lstlisting}
>>> 2+2
>>> "Bonjour le monde !"
\end{lstlisting}

Voici quelques instructions.
\begin{itemize}
  \item \textbf{Addition.} \ci{5+7}.
  \item \textbf{Multiplication.} 
  \ci{6*7} ; avec des parenthèses \ci{3*(12+5)} ; avec des nombres à virgule \ci{3*1.5}.
  \item \textbf{Puissance.}\index{puissance}\index{**@\ci{**}} \ci{3**2} pour $3^2=9$ ; puissance négative \ci{10**-3} pour $10^{-3} = 0.001$.
  \item \textbf{Division réelle.} \ci{14/4} vaut \ci{3.5} ; \ci{1/3} vaut \ci{0.3333333333333333}.
  \item \textbf{Division entière\index{division euclidienne} et modulo\index{modulo}\index{reste}.}
  \begin{itemize}
    \item \ci{14//4} vaut \ci{3} : c'est le quotient de la division euclidienne de $14$ par $4$, note bien la double barre ;
    \item \ci{14\%4} vaut \ci{2} : c'est le reste de la division euclidienne de $14$ par $4$, on dit aussi \og{}$14$ modulo $4$\fg.
  \end{itemize}
\end{itemize}

\emph{Remarque.} Dans tout ce cours, on écrira les \og{}nombres à virgule\fg{} sous la forme $3.5$ (et pas $3,5$). Le séparateur décimal est donc le point. En informatique les nombres à virgule sont appelés \og{}nombres flottants\fg{}\index{nombre flottant}.
\end{cours}


%%%%%%%%%%%%%%%%%%%%%%%%%%%%%%%%%%%%%%%%%%%%%%%%%%%%%%%%%%%%%%%%
% Activité 1
%%%%%%%%%%%%%%%%%%%%%%%%%%%%%%%%%%%%%%%%%%%%%%%%%%%%%%%%%%%%%%%%

\begin{activite}[Premiers pas]

\objectifs{Objectifs : faire tes premiers calculs avec \Python.}

\begin{enumerate}
  \item Combien y a-t-il de secondes en un siècle ? (Ne tiens pas compte des années bissextiles.)
  
  \item Jusqu'où faut-il compléter les pointillés pour obtenir un nombre plus grand qu'un milliard ?
  $$(1+2)\times(3+4)\times(5+6)\times(7+8)\times\cdots$$
  
  \item Quels sont les trois derniers chiffres de 
  $$\underbrace{123456789 \times 123456789 \times \cdots}_{7 \text{ occurrences de } 123456789} \quad ?$$
  
  \item $7$ est le premier entier tel que son inverse a une écriture décimale périodique de longueur $6$ :
  $$\frac{1}{7} = 0.\underbrace{142857}\underbrace{142857}\underbrace{142857}\ldots$$ Trouve le premier entier dont l'inverse a une écriture décimale périodique de longueur $7$ :
  $$\frac{1}{???} = 0.00\underbrace{abcdefg}\underbrace{abcdefg}\ldots$$
  
  \emph{Indication.} L'entier est plus grand que $230$ !
  
  \item Trouve l'unique entier :
  \begin{itemize}
    \item qui donne un quotient de $107$ lorsque l'on effectue sa division (euclidienne) par $11$,
    \item et qui donne un quotient de $90$ lorsque l'on effectue sa division (euclidienne) par $13$,
    \item et qui donne un reste égal à $6$ modulo $7$ !
  \end{itemize}
    
\end{enumerate}

\end{activite}

%%%%%%%%%%%%%%%%%%%%%%%%%%%%%%%%%%%%%%%%%%%%%%%%%%%%%%%%%%%%%%%%
%%%%%%%%%%%%%%%%%%%%%%%%%%%%%%%%%%%%%%%%%%%%%%%%%%%%%%%%%%%%%%%%

\begin{cours}[Travailler avec un éditeur]
À partir de maintenant, il est préférable que tu travailles dans un éditeur de texte dédié à \Python. Tu dois alors explicitement demander d'afficher\index{print@\ci{print}} le résultat :
\begin{lstlisting}
print(2+2)
print("Bonjour le monde !")
\end{lstlisting}

%Dans la suite tu continues d'écrire ton code dans l'éditeur mais on n'indiquera plus qu'il faut utiliser \ci{print()} pour afficher les résultats.
\end{cours}



%%%%%%%%%%%%%%%%%%%%%%%%%%%%%%%%%%%%%%%%%%%%%%%%%%%%%%%%%%%%%%%%
%%%%%%%%%%%%%%%%%%%%%%%%%%%%%%%%%%%%%%%%%%%%%%%%%%%%%%%%%%%%%%%%

\begin{cours}[Variables]


\textbf{Variable.} 
Une \defi{variable}\index{variable} est un nom associé à un emplacement de la mémoire. C'est comme une boîte que l'on identifie par une étiquette.
La commande \og{}\ci{a = 3}\fg{}\index{=@\ci{=}}\index{affectation} signifie que j'ai une variable \og{}\ci{a}\fg{} associée à la valeur $3$. 

Voici un premier exemple :
\begin{lstlisting}
a = 3  # Une variable
b = 5  # Une autre variable

print("La somme vaut",a+b)   # Affiche la somme
print("Le produit vaut",a*b) # Affiche le produit

c = b**a     # Nouvelle variable...
print(c)     # ... qui s'affiche
\end{lstlisting}


\medskip

\textbf{Commentaires.}
\index{\#@\#}
\index{commentaire}
Tout texte qui suit le caractère dièse \og{}\#\fg{} n'est pas exécuté par \Python{} mais sert à expliquer le programme. C'est une bonne habitude de commenter abondamment ton code.

 
\medskip

\textbf{Noms.} 
Il est très important de donner un nom clair et précis aux variables. Par exemple, avec les noms bien choisis tu devrais savoir ce que calcule le code suivant :
\begin{lstlisting}
base = 8
hauteur = 3
aire = base * hauteur / 2
print(aire)
print(Aire)  # !! Erreur !!
\end{lstlisting}

Attention ! \Python{} distingue les majuscules des minuscules. Donc \ci{mavariable}, \ci{Mavariable} et \ci{MAVARIABLE} sont des variables différentes.

\medskip

\textbf{Réaffectation.}\index{affectation} Imaginons que tu veuilles tenir tes comptes journaliers. Tu pars d'une somme $S_0 = 1000$, le lendemain tu gagnes $100$, donc maintenant $S_1 = S_0 + 100$ ; le jour d'après tu rajoutes $200$, donc $S_2 = S_1 + 200$ ; puis tu perds $50$, donc au troisième jour $S_3 = S_2 - 50$. Avec \Python{} tu peux n'utiliser qu'une seule variable \ci{S} pour toutes ces opérations.

\begin{lstlisting}
S = 1000
S = S + 100
S = S + 200
S = S - 50
print(S)
\end{lstlisting}
Il faut comprendre l'instruction \og{}\ci{S = S + 100}\fg{} comme ceci : \og{}je prends le contenu de la boîte \ci{S}, je rajoute $100$, je remets tout dans la même boîte\fg{}.
\end{cours}



%%%%%%%%%%%%%%%%%%%%%%%%%%%%%%%%%%%%%%%%%%%%%%%%%%%%%%%%%%%%%%%%
% Activité 2
%%%%%%%%%%%%%%%%%%%%%%%%%%%%%%%%%%%%%%%%%%%%%%%%%%%%%%%%%%%%%%%%

\begin{activite}[Variables]

\objectifs{Objectifs : utiliser des variables !}

\begin{enumerate}
  \item 
  \begin{enumerate}
    \item   Définis des variables, puis calcule l'aire d'un trapèze. 
  Ton programme doit afficher \ci{"L'aire vaut ..."} en utilisant \ci{print("L'aire vaut",aire)}.

\myfigure{0.9}{
  \tikzinput{fig-trapeze}
}  

    
    \item Définis des variables pour calculer le volume d'une boîte (un parallélépipède rectangle) dont les dimensions sont $10$, $8$, $3$.
    
    \item Définis une variable \ci{PI} qui vaut $3.14$. Définis un rayon $R = 10$. Écris la formule de l'aire du disque de rayon $R$.

\end{enumerate}    

  
  \item Remets les lignes dans l'ordre de sorte qu'à la fin $x$ ait la valeur $46$.
\begin{center}
\begin{minipage}{0.5\textwidth}
\begin{lstlisting}  
(1)  y = y - 1
(2)  y = 2*x
(3)  x = x + 3*y     
(4)  x = 7
\end{lstlisting}
\end{minipage}
\end{center}  
  
  \item Tu places la somme de $1000$ euros sur un compte d'épargne. Chaque année les intérêts sur l'argent placé rapportent $10\%$ (le capital est donc multiplié par $1.10$).
  Écris le code qui permet de calculer le capital pour les trois premières années.


  \item Je définis deux variables par \ci{a = 9} et \ci{b = 11}. Je souhaite échanger le contenu de \ci{a} et \ci{b}. Quelles instructions conviennent de sorte qu'à la fin \ci{a} vaut $11$ et \ci{b} vaut $9$ ?
  
\begin{center}
\begin{minipage}{0.2\textwidth}
\begin{lstlisting}  
a = b
b = a   
\end{lstlisting}
\end{minipage}
\begin{minipage}{0.2\textwidth}
\begin{lstlisting}  
c = b
a = b
b = c
\end{lstlisting}
\end{minipage}
\begin{minipage}{0.2\textwidth}
\begin{lstlisting}
c = a
a = b
b = c 
\end{lstlisting}
\end{minipage}
\begin{minipage}{0.2\textwidth}
\begin{lstlisting} 
c = a
a = c
c = b
b = c
\end{lstlisting}
\end{minipage}

\end{center} 
 
\end{enumerate}  
\end{activite}


%%%%%%%%%%%%%%%%%%%%%%%%%%%%%%%%%%%%%%%%%%%%%%%%%%%%%%%%%%%%%%%%
%%%%%%%%%%%%%%%%%%%%%%%%%%%%%%%%%%%%%%%%%%%%%%%%%%%%%%%%%%%%%%%%

\begin{cours}[Utiliser des fonctions]
\sauteligne
\begin{itemize}
  \item \textbf{Utiliser des fonctions de \Python.}
  
  Tu connais déjà la fonction \ci{print()} qui affiche une chaîne de caractères (ou des nombres). Elle s'utilise ainsi \ci{print("Coucou")} ou bien à travers une valeur : 
\begin{center}
\begin{minipage}{0.5\textwidth}
\begin{lstlisting}  
chaine = "Bonjour"
print(chaine)
\end{lstlisting}
\end{minipage}
\end{center} 	 

  Il existe plein d'autres fonctions. Par exemple la fonction \ci{abs()} calcule la valeur absolue : \ci{abs(-3)} renvoie \ci{3}, \ci{abs(5)} renvoie \ci{5}.
 % Autre exemple : \ci{float()} transforme une chaîne de caractères en un nombre : \ci{float("+1.234567")} renvoie le nombre flottant \ci{1.234567}.

  \item \textbf{Le module \ci{math}.}

	Toutes les fonctions ne sont pas directement accessibles. Elles sont souvent regroupées dans des \defi{modules}\index{module}. Par exemple le module \ci{math}\index{math@\ci{math}}\index{module!math@\ci{math}} contient les fonctions mathématiques. Tu y trouves par exemple la fonction racine carrée \ci{sqrt()}\index{sqrt@\ci{sqrt}}\index{racine carree@racine carrée} (\emph{square root}). Voici comment l'utiliser : 
\begin{center}
\begin{minipage}{0.5\textwidth}
\begin{lstlisting} 
from math import *

x = sqrt(2)
print(x)
print(x**2)
\end{lstlisting}
\end{minipage}

\end{center} 	

La première ligne importe toutes les fonctions du module \ci{math}\index{import@\ci{import}}, la seconde calcule $x = \sqrt{2}$ (en valeur approchée) et ensuite on affiche $x$ et $x^2$.	
  \item \textbf{Sinus et cosinus.} 
  
  Le module \ci{math} contient les fonctions trigonométriques sinus\index{sin@\ci{sin}} et cosinus\index{cos@\ci{cos}} et même la constante \ci{pi}\index{pi@\ci{pi}}\index{$\pi$} qui est une valeur approchée de $\pi$. Attention, les angles sont exprimés en radians.

Voici le calcul de $\sin(\frac\pi2)$. 
\begin{center}
\begin{minipage}{0.5\textwidth} 
\begin{lstlisting}
angle = pi/2
print(angle)
print(sin(angle))
\end{lstlisting}
\end{minipage}
\end{center}

  \item \textbf{Décimal vers entier.}

Dans le module \ci{math} il y aussi des fonctions pour arrondir un nombre décimal : 
  \begin{itemize}
    \item \ci{round()}\index{round@\ci{round}} arrondit à l'entier le plus proche : \ci{round(5.6)} renvoie \ci{6}, \ci{round(1.5)} renvoie \ci{2}.
    \item \ci{floor()}\index{floor@\ci{floor}} renvoie l'entier inférieur ou égal : \ci{floor(5.6)} renvoie \ci{5}.
    \item \ci{ceil()}\index{ceil@\ci{ceil}} renvoie l'entier supérieur ou égal : \ci{ceil(5.6)} renvoie \ci{6}.
  \end{itemize}       
    
\end{itemize}

\end{cours}

%%%%%%%%%%%%%%%%%%%%%%%%%%%%%%%%%%%%%%%%%%%%%%%%%%%%%%%%%%%%%%%%
% Activité 3
%%%%%%%%%%%%%%%%%%%%%%%%%%%%%%%%%%%%%%%%%%%%%%%%%%%%%%%%%%%%%%%%

\begin{activite}[Utiliser des fonctions]

\objectifs{Objectifs : utiliser des fonctions de \Python{} et du module \ci{math}.}

\begin{enumerate}
  \item  La fonction \Python{} pour le pgcd est \ci{gcd(a,b)}\index{gcd@\ci{gcd}}\index{pgcd} (sans le \og{}p\fg{}, pour \emph{greatest common divisor}). Calcule le pgcd de $a = 10\,403$ et $b = 10\,506$. Déduis-en le ppcm\index{ppcm} de $a$ et $b$. La fonction ppcm n'existe pas, tu dois utiliser la formule :
  $$\text{ppcm}(a,b) = \frac{a \times b}{\text{pgcd}(a,b)}.$$
  
  \item Trouve par tâtonnement un nombre réel $x$ qui vérifie toutes les conditions suivantes (plusieurs solutions sont possibles) :
  \begin{itemize}
    \item \ci{abs(x**2 - 15)} est inférieur à \ci{0.5}
    \item \ci{round(2*x)} renvoie \ci{8}
    \item \ci{floor(3*x)} renvoie \ci{11}
    \item \ci{ceil(4*x)} renvoie \ci{16} 
  \end{itemize}
 
  \emph{Indication.} \ci{abs()}\index{abs@\ci{abs}}\index{valeur absolue} désigne la fonction valeur absolue.
  
  \item Tu connais la formule de trigonométrie 
  $$\cos^2 \theta + \sin^2 \theta = 1.$$
  Vérifie que pour $\theta = \frac\pi7$ (ou d'autres valeurs) cette formule est numériquement vraie. (Ce n'est pas une preuve de la formule, car \Python{} ne fait que des calculs approchés du sinus et du cosinus).
\end{enumerate}  
\end{activite}

%%%%%%%%%%%%%%%%%%%%%%%%%%%%%%%%%%%%%%%%%%%%%%%%%%%%%%%%%%%%%%%%
%%%%%%%%%%%%%%%%%%%%%%%%%%%%%%%%%%%%%%%%%%%%%%%%%%%%%%%%%%%%%%%%

\begin{cours}[Boucle \og{}pour\fg{}]

\objectifs{La boucle \og{}pour\fg{} est la façon la plus simple de répéter des instructions.}

\index{for@\ci{for}}
\index{in@\ci{in}}
\index{boucle!pour}

\mybox{
\myfigure{0.7}{
  \tikzinput{fig-premiers_pas-boucle-pour}
} }


Note bien que ce qui délimite le bloc d'instructions à répéter, c'est \defi{l'indentation}\index{indentation}, c'est-à-dire les espaces placées en début de ligne qui décalent les lignes vers la droite.
Toutes les lignes d'un bloc doivent avoir exactement la même indentation. Dans ce livre, nous choisissons une indentation de $4$ espaces.

N'oublie pas les deux points en fin de la ligne de la déclaration du \ci{for} !

\begin{itemize}
  \item \textbf{Exemple de boucle \og{}pour\fg{}.}

Voici une boucle qui affiche les premiers carrés. 
\begin{center}
\begin{minipage}{0.5\textwidth} 
\begin{lstlisting}
for i in range(10):
    print(i*i)
\end{lstlisting}
\end{minipage}
\end{center} 
La seconde ligne est décalée et constitue le bloc à répéter.
La variable \ci{i} prend la valeur $0$ et l'instruction affiche $0^2$ ;
puis \ci{i} prend la valeur $1$, et l'instruction affiche $1^2$ ; puis $2^2$, $3^2$\ldots

Au final ce programme affiche :
$$0,1,4,9,16,25,36,49,64,81.$$

Attention : la dernière valeur prise par \ci{i} est bien $9$ (et pas $10$).

  \item \textbf{Parcourir une liste quelconque.}
  
La boucle \og{}pour\fg{} permet de parcourir n'importe quelle liste. Voici une boucle qui affiche le cube des premiers nombres premiers.
\begin{center}
\begin{minipage}{0.5\textwidth} 
\begin{lstlisting}
for p in [2,3,5,7,11,13]:
    print(p**3)
\end{lstlisting}
\end{minipage}
\end{center} 

  \item \textbf{Sommes des entiers.}

Voici un programme qui calcule 
$$0+1+2+3+\cdots + 18 +19.$$

\begin{center}
\begin{minipage}{0.5\textwidth} 
\begin{lstlisting}
somme = 0
for i in range(20):
    somme = somme + i
print(somme)
\end{lstlisting}
\end{minipage}
\end{center} 

Comprends bien ce code : une variable \ci{somme} est initialisée à $0$. 
On va tour à tour lui ajouter $0$, puis $1$, puis $2$\ldots{}
On peut mieux comprendre cette boucle en complétant un tableau :
  $$
  \begin{array}{l}
  \text{Initialisation : \ci{somme}$=0$}    \\
  \begin{array}{|c|c|}
  \hline  
   \text{\ci{i}} &  \text{\ci{somme}} \\
  \hline\hline
  0 & 0 \\  
  1 & 1 \\
  2 & 3 \\
  3 & 6 \\
  4 & 10\\
  \hspace*{5ex}\ldots\hspace*{5ex} & \hspace*{5ex}\ldots\hspace*{5ex} \\
  18 & 171 \\
  19 & 190 \\ 
  \hline
  \end{array} \\
  \text{Affichage : $190$}  
  \end{array} 
  $$ 
\end{itemize} 

% pas d'index dans une mybox !!  
\index{range@\ci{range}}
\index{list@\ci{list}}  
\mybox
{ 
 \begin{minipage}{0.95\textwidth}
 \begin{itemize}
 \item \textbf{\ci{range()}.}
	\begin{itemize}
	  \item Avec \ci{range(n)} on parcourt les entiers de $0$ à $n-1$.
	  Par exemple \ci{range(10)} correspond à la liste \ci{[0, 1, 2, 3, 4, 5, 6, 7, 8, 9]}. 
	  
	  Attention ! la liste s'arrête bien à $n-1$ et pas à $n$. Ce qu'il faut retenir 
	  c'est que la liste contient bien $n$ éléments (car elle commence à $0$).
	  
	 \item Si tu veux afficher la liste des éléments parcourus, il faut utiliser la commande :\\
	  \centerline{\ci{list(range(10))}}
	
	\item Avec \ci{range(a,b)} on parcourt les éléments de $a$ à $b-1$.
	Par exemple  \ci{range(10,20)} correspond à la liste \ci{[10, 11, 12, 13, 14, 15, 16, 17, 18, 19]}.  
	
	\item Avec \ci{range(a,b,pas)} on parcourt les éléments $a$, $a+\text{pas}$, $a + 2\text{pas}$\ldots{} Par exemple \ci{range(10,20,2)} correspond à la liste \ci{[10, 12, 14, 16, 18]}.  

	\end{itemize}
\end{itemize}
\end{minipage}	
}
\begin{itemize}
  \item \textbf{Imbrication de boucles.}

Il est possible d'imbriquer des boucles, c'est-à-dire que dans le bloc d'une boucle, on utilise une nouvelle boucle.
\begin{center}
\begin{minipage}{0.5\textwidth} 
\begin{lstlisting}
for x in [10,20,30,40,50]:
    for y in [3,7]:
        print(x+y)
\end{lstlisting}
\end{minipage}
\end{center}
Dans ce petit programme $x$ vaut d'abord $10$, $y$ prend la valeur $3$ puis la valeur $7$ (le programme affiche donc $13$, puis $17$). Ensuite $x=20$, et $y$ vaut de nouveau $3$ puis $7$ (le programme affiche donc ensuite $23$, puis $27$). 
Au final le programme affiche :
$$13,17,23,27,33,37,43,47,53,57.$$
    
\end{itemize} 
\end{cours}


%%%%%%%%%%%%%%%%%%%%%%%%%%%%%%%%%%%%%%%%%%%%%%%%%%%%%%%%%%%%%%%%
% Activité 4
%%%%%%%%%%%%%%%%%%%%%%%%%%%%%%%%%%%%%%%%%%%%%%%%%%%%%%%%%%%%%%%%

\begin{activite}[Boucle \og{}pour\fg{}]

\objectifs{Objectifs : construire des boucles simples.}

\begin{enumerate}
  \item 
  \begin{enumerate}
    \item Affiche les cubes des entiers de $0$ à $100$.
    \item Affiche les puissances quatrièmes des entiers de $10$ à $20$.
    \item Affiche les racines carrées des entiers $0$, $5$, $10$, $15$,\ldots{} jusqu'à $100$.   
  \end{enumerate}

  \item Affiche les puissances de $2$, de $2^1$ à $2^{10}$, et apprends par c\oe ur les résultats !
  
  
  \item Recherche de façon expérimentale une valeur approchée du minimum de la fonction 
  $$f(x) = x^3-x^2-\frac14x+1$$
   sur l'intervalle $[0,1]$.
  
  \emph{Indications.}
  \begin{itemize}
    \item Construis une boucle dans laquelle une variable $i$ balaye les entiers de $0$ à $100$.
    \item Définis $x=\frac{i}{100}$. Ainsi $x=0.00$, puis $x=0.01$, $x=0.02$\ldots
    \item Calcule $y = x^3-x^2-\frac14x+1$.
    \item Affiche les valeurs à l'aide de \ci{print("x =",x,"y =",y)}.
    \item Cherche à la main pour quelle valeur de $x$ on obtient un $y$ le plus petit possible.
    \item N'hésite pas à modifier ton programme pour augmenter la précision.
  \end{itemize}
    
  \item  Cherche une valeur approchée que doit avoir
  le rayon $R$ d'une boule afin que son volume soit $100$ ?
  
  \emph{Indications.} 
  \begin{itemize}
    \item Utilise une méthode de balayage comme à la question précédente.
    \item La formule du volume d'une boule est $V = \frac43 \pi R^3$.
    \item Affiche les valeurs à l'aide de \ci{print("R =",R,"V =",V)}.    
    \item Pour $\pi$ tu peux prendre la valeur approchée $3.14$ ou bien la valeur approchée \ci{pi} du module \ci{math}.
  \end{itemize}
  
\end{enumerate}  
\end{activite}


%%%%%%%%%%%%%%%%%%%%%%%%%%%%%%%%%%%%%%%%%%%%%%%%%%%%%%%%%%%%%%%%
% Activité 5
%%%%%%%%%%%%%%%%%%%%%%%%%%%%%%%%%%%%%%%%%%%%%%%%%%%%%%%%%%%%%%%%

\begin{activite}[Boucle \og{}pour\fg{} (suite)]

\objectifs{Objectifs : construire des boucles plus compliquées.}

\begin{enumerate}
  \item Définis une variable $n$ (par exemple $n=20$). Calcule la somme
  $$1^2+2^2+3^2+\cdots+i^2+\cdots +n^2.$$
  
  \item Calcule le produit :
  $$1 \times 3 \times 5 \times \cdots \times 19.$$
  
  \emph{Indications.} Commence par définir une variable \ci{produit} initialisée à la valeur $1$. Utilise \ci{range(a,b,2)} pour obtenir un entier sur deux.
  
  \item Affiche les tables de multiplication entre $1$ et $10$. Voici un exemple de ligne à afficher :\\
  \centerline{\ci{7 x 9 = 63}}
  Utilise une commande d'affichage du style : \ci{print(a,"x",b,"=",a*b)}.
\end{enumerate}  
\end{activite}

\end{document}
