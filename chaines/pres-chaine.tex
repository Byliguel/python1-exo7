\documentclass[12pt,class=report,crop=false]{standalone}
\usepackage[screen]{../python}

\pagestyle{empty}

\begin{document}

%====================================================================
\chapitre{Chaînes de caractères -- Analyse d'un texte}
%====================================================================

\section*{Caractère et chaîne}


\begin{itemize}
  \item Un \defi{caractère}\index{caractere@caractère} est un symbole unique, par exemple une lettre minuscule \codeinline{"a"}, une lettre majuscule \codeinline{"B"}, un symbole spécial \codeinline{"\&"}, un symbole représentant un chiffre \codeinline{"7"}, une espace \codeinline{" "} que l'on notera aussi \lstinline[showstringspaces=true]!" "!.
  
  \bigskip
  
  \item 
Pour désigner un caractère, il faut le mettre entre guillemets simples \codeinline{'z'} ou entre guillemets doubles \codeinline{"z"}.

    \bigskip
    
  \item Une \defi{chaîne de caractères}\index{chaine@chaîne} est une suite de caractères, comme un mot \codeinline{"Bonjour"}, une phrase \codeinline{'Il fait beau.'}, un mot de passe \codeinline{"N[w5ms\}e!"}.
  
    \bigskip
  
  \item Le type d'un caractère ou d'une chaîne est \codeinline{str}\index{str@\ci{str}} (pour \emph{string}).
  

\end{itemize}



\newpage

\section*{Opérations sur les chaînes}


\begin{itemize}
  \item La \defi{concaténation}\index{concatenation@concaténation}, c'est-à-dire la mise bout à bout de deux chaînes, s'effectue à l'aide de l'opérateur \codeinline{+}. Par exemple \ci{"para"+"pluie"} donne la chaîne \ci{"parapluie"}.
 
   \bigskip
 
  \item La chaîne vide \ci{""} est utile lorsque l'on veut initialiser une chaîne avant d'y ajouter d'autres caractères.
  
    \bigskip
  
  \item La \defi{longueur} d'une chaîne est le nombre de caractères qu'elle contient. Elle s'obtient par la fonction \ci{len()}\index{len@\ci{len}}.
  Par exemple \lstinline[showstringspaces=true]!len("Hello World")! renvoie $11$ (l'espace compte comme un caractère).
\end{itemize}  

\newpage  
 
\section*{Opérations sur les chaînes}
 
Si \ci{mot} est une chaîne alors on peut récupérer chaque caractère par \ci{mot[i]}. Par exemple si 
  \ci{mot = "avion"} alors :
  \begin{itemize}
    \item \ci{mot[0]} est le caractère \ci{"a"},
    \item \ci{mot[1]} est le caractère \ci{"v"}, 
    \item \ci{mot[2]} est le caractère \ci{"i"},
    \item \ci{mot[3]} est le caractère \ci{"o"},       
    \item \ci{mot[4]} est le caractère \ci{"n"}.
  \end{itemize} 
  
\bigskip  
  
\begin{center}
\begin{tabular}{|c||c|c|c|c|c|}
\hline
Lettre & a & v & i & o & n \\ \hline
Rang & 0 & 1 & 2 & 3 & 4 \\ \hline
\end{tabular}
\end{center}


\bigskip

Si \ci{mot} est une chaîne, les caractères s'obtiennent par \ci{mot[i]} pour \ci{i} variant de \ci{0} à \ci{len(mot)-1}.


%%%%%%%%%%%%%%%%%%%%%%%%%%%%%%%%%%%%%%%%%%%%%%%%%%%%%%%%%%%%%%%%
%%%%%%%%%%%%%%%%%%%%%%%%%%%%%%%%%%%%%%%%%%%%%%%%%%%%%%%%%%%%%%%%


\newpage

\section*{Sous-chaînes}


On peut extraire plusieurs caractères d'une chaîne à l'aide de la syntaxe \ci{mot[i:j]} qui renvoie une chaîne formée des caractères numéro $i$ à $j-1$ (attention le caractère numéro $j$ n'est pas inclus !).

    \bigskip

Par exemple si \ci{mot = "vendredi"} alors :
\begin{itemize}
  \item \ci{mot[0:4]} renvoie la sous-chaîne \ci{"vend"} formée des caractères de rang $0$, $1$, $2$ et $3$ (mais pas $4$),
  
      \bigskip
      
  \item \ci{mot[3:6]} renvoie \ci{"dre"} correspondant aux rangs $3$, $4$ et $5$.
\end{itemize}  

    \bigskip

\begin{center}
\begin{tabular}{|c||c|c|c|c|c|c|c|c|}
\hline
Lettre & v & e & n & d & r & e & d & i \\ \hline
Rang   & 0 & 1 & 2 & 3 & 4 & 5 & 6 & 7\\ \hline
\end{tabular}
\end{center}
 
     \bigskip
     
Autre exemple : \ci{mot[1:len(mot)-1]} renvoie le mot privé de sa première et dernière lettre.


\newpage


\section*{Un peu plus sur les chaînes}


\begin{itemize}
  \item Une boucle \ci{for ... in ...} permet de parcourir une chaîne, caractère par caractère :\index{boucle!pour}\index{for@\ci{for}}
  \begin{center}
  \begin{minipage}{0.4\textwidth}
  \ci{for carac in mot:}\\
  \indentation \ci{print(carac)}
  \end{minipage}
  \end{center}

  \bigskip

  \item On peut tester si un caractère appartient à une certaine liste de caractères. Par exemple : \\  
  \centerline{\codeinline{if carac in ["a", "A", "b", "B", "c", "C"]:}}
\index{in@\ci{in}}  
  
  permet d’exécuter des instructions si le caractère \ci{carac}  est l'une des lettres a, A, b, B, c, C. 
  
    \bigskip
    
    \item 
  Pour éviter certaines lettres, on utiliserait : \\
  \centerline{\codeinline{if carac not in ["X", "Y", "Z"]:}}
  \index{not in@\ci{not in}}

  \end{itemize}  

\newpage

\section*{Codage des caractères}

\index{codage des caracteres@codage des caractères}

Un caractère est stocké par l'ordinateur sous la forme d'un entier.
Pour le codage ASCII/unicode, la lettre majuscule \og{}A\fg{} est codé par $65$, la lettre minuscule \og{}h\fg{} est codée par $104$, le symbole \og{}\#\fg{} par $35$.

Voici la table des premiers caractères. Les numéros $0$ à $32$ ne sont pas des caractères imprimables. Cependant le numéro $32$ est le caractère espace \lstinline!" "!.

\index{ascii}
\index{unicode}

\myfigure{0.8}{
\small
  \tikzinput{chaines-unicode}
} 


\newpage

\section*{Codage des caractères}

La fonction \ci{chr()}\index{chr@\ci{chr}} est une fonction \Python{} qui renvoie le caractère associé à un code.
  
  \begin{fonctionpython}[\ci{python : chr()}]
  Usage : \ci{chr(code)}\\
   Entrée : un code (un entier)\\
   Sortie : un caractère
  
  \medskip
     
   Exemple :
  \begin{itemize}  
    \item \ci{chr(65)} renvoie \ci{"A"}
    \item \ci{chr(123)} renvoie \ci{"\{"}
  \end{itemize} 
  \end{fonctionpython} 
  
\bigskip


La fonction \ci{ord()}\index{ord@\ci{ord}} est une fonction \Python{} correspondant à l'opération inverse : elle renvoie le code associé à un caractère.
  
  \begin{fonctionpython}[\ci{python : ord()}]
  Usage : \ci{ord(carac)}\\
   Entrée : un caractère (une chaîne de longueur $1$)\\
   Sortie : un entier
  
  \medskip
     
   Exemple :
  \begin{itemize}  
    \item \ci{ord("A")} renvoie \ci{65}
    \item \ci{ord("*")} renvoie \ci{42}
  \end{itemize} 
  \end{fonctionpython}  


\end{document}
