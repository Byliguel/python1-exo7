\documentclass[11pt,class=report,crop=false]{standalone}
\usepackage[screen]{../python}
\begin{document}

% Commande spécifique
\newcommand{\badletter}[1]{\underline{\textcolor{red}{#1}}}



%====================================================================
\chapitre{Binaire I}
%====================================================================

\objectifs{Les ordinateurs transforment toutes les données en nombres et manipulent uniquement ces nombres. Ces nombres sont stockés sous la forme de listes de $0$ et de $1$. C'est l'écriture binaire des nombres !
Pour mieux comprendre l'écriture binaire, tu vas d'abord mieux comprendre l'écriture décimale.
}

\index{binaire}
\index{ecriture@écriture!binaire}


\insertvideo{EhxiXresX7Y}{Binaire - partie 1 - écriture décimale}

\insertvideo{_NrAp2C16j4}{Binaire - partie 2 - de l'écriture binaire vers l'entier}

\insertvideo{D9ZSxpCvE7A}{Binaire - partie 3 - de l'entier vers l'écriture binaire}


%%%%%%%%%%%%%%%%%%%%%%%%%%%%%%%%%%%%%%%%%%%%%%%%%%%%%%%%%%%%%%%%
%%%%%%%%%%%%%%%%%%%%%%%%%%%%%%%%%%%%%%%%%%%%%%%%%%%%%%%%%%%%%%%%

\begin{cours}[\'Ecriture décimale]

\index{ecriture@écriture!decimale@décimale}

L'écriture habituelle des entiers se fait dans le système décimal (en base $10$). Par exemple, $70\,685$ c'est 
${\color{blue}7} \times 10\,000
+{\color{blue}0} \times 1\,000
+{\color{blue}6} \times 100
+{\color{blue}8} \times 10 
+{\color{blue}5} \times 1$ : 
$$
\begin{array}{|c|c|c|c|c|}
  \hline
  {\color{blue}7} & {\color{blue}0} & {\color{blue}6} & {\color{blue}8} & {\color{blue}5} \\ 
  \hline\hline
  10000 & 1000 & 100 & 10 & 1 \\\hline
  10^4 & 10^3 & 10^2 & 10^1 & 10^0 \\
  \hline
\end{array}
$$   
(on voit bien que $5$ est le chiffre des unités, $8$ celui des dizaines, $6$ celui des centaines\ldots). 

Il faut bien comprendre les puissances de $10$. On note $10^k$ pour $10 \times 10 \times \cdots \times 10$ (avec $k$ facteurs).

$$
\begin{array}{|c|c|c|c|c|c|c|c|}
  \hline
  {\color{blue}d_{p-1}} & {\color{blue}d_{p-2}}& {\color{blue}\cdots} & {\color{blue}d_i}& {\color{blue}\cdots} & {\color{blue}d_2} & {\color{blue}d_1} & {\color{blue}d_0} \\ 
  \hline\hline
  10^{p-1} & 10^{p-2} & \cdots & 10^i &\cdots & 10^2 & 10^1 & 10^0 \\
  \hline
\end{array}
$$   
  
On calcule donc l'entier correspondant aux chiffres $[d_{p-1},d_{p-2}, \ldots,d_2,d_1,d_0]$ par la formule :
$$n = {\color{blue}d_{p-1}} \times 10^{p-1} + {\color{blue}d_{p-2}} \times 10^{p-2} + \cdots + {\color{blue}d_i} 10^{i} +  \cdots + {\color{blue}d_2} \times 10^2 + {\color{blue}d_1} \times 10^1 + {\color{blue}d_0} \times 10^0$$
\end{cours}

%%%%%%%%%%%%%%%%%%%%%%%%%%%%%%%%%%%%%%%%%%%%%%%%%%%%%%%%%%%%%%%%
% Activité 1
%%%%%%%%%%%%%%%%%%%%%%%%%%%%%%%%%%%%%%%%%%%%%%%%%%%%%%%%%%%%%%%%


\begin{activite}[De l'écriture décimale vers l'entier]

\objectifs{Objectifs : à partir de l'écriture décimale, retrouver l'entier.}

\'Ecris une fonction \ci{decimale_vers_entier(liste_decimale)} qui à partir d'une liste représentant l'écriture décimale calcule l'entier correspondant.
  
  \begin{fonction}[\ci{decimale_vers_entier()}]
  Usage : \ci{decimale_vers_entier(liste_decimale)} \\
  Entrée : une liste de chiffres entre $0$ et $9$ \\
  Sortie : l'entier dont l'écriture décimale est la liste
  
  \medskip
  Exemple : si l'entrée est \ci{[1,2,3,4]}, la sortie est \ci{1234}.
  \end{fonction} 
  
  \emph{Indications.}  Il faut faire la somme d'éléments du type
    $$d_i \times 10^{i} \quad \text{ pour } 0 \le i < p$$
    où $p$ est la longueur de la liste et $d_i$ est le chiffre en position $i$ \emph{en comptant à partir de la fin} (c'est-à-dire de droite à gauche). Pour gérer le fait que l'indice $i$ utilisé pour parcourir la liste ne correspond pas à la puissance de $10$, il y a deux solutions :
  \begin{itemize}
    \item comprendre que $d_i = $\ci{liste}$[p-1-i]$ où \ci{liste} est la liste des $p$ chiffres,
    \item ou bien commencer par inverser \ci{liste}.    
   \end{itemize}  
  
\end{activite}

%%%%%%%%%%%%%%%%%%%%%%%%%%%%%%%%%%%%%%%%%%%%%%%%%%%%%%%%%%%%%%%%
%%%%%%%%%%%%%%%%%%%%%%%%%%%%%%%%%%%%%%%%%%%%%%%%%%%%%%%%%%%%%%%%


\begin{cours}[Binaire]
\sauteligne
\begin{itemize}
  \item \textbf{Puissances de 2.}
    On note $2^k$ pour $2 \times 2 \times \cdots \times 2$ (avec $k$ facteurs). Par exemple, $2^3 = 2 \times 2 \times 2 = 8$. 
$$
\begin{array}{|c|c|c|c|c|c|c|c|}
  \hline
  2^7  & 2^6 & 2^5 & 2^4 & 2^3 & 2^2 & 2^1 & 2^0 \\
  \hline
  128 & 64 & 32 & 16 & 8 & 4 & 2 & 1 \\ 
  \hline
\end{array}
$$
\index{puissances de 2}

  \item \textbf{\'Ecriture binaire : un exemple.}
  
    Tout entier admet une écriture binaire, c'est-à-dire une écriture où les seuls chiffres sont des $0$ ou des $1$. En binaire, les chiffres sont appelés des \defi{bits}.\index{bits@\emph{bits}}
     Par exemple, $1.0.1.1.0.0.1$ (prononce les chiffres un par un) est l'écriture binaire de l'entier $89$. Comment faire ce calcul ? C'est comme pour la base $10$, mais en utilisant les puissances de $2$ ! 
$$
\begin{array}{|c|c|c|c|c|c|c|}
  \hline
  {\color{red}1} & {\color{red}0} & {\color{red}1} & {\color{red}1} & {\color{red}0} & {\color{red}0} & {\color{red}1} \\ 
  \hline\hline
  64 & 32 & 16  & 8 & 4 & 2 & 1 \\\hline
  2^6 & 2^5 & 2^4 & 2^3 & 2^2 & 2^1 & 2^0 \\
  \hline
\end{array}
$$


    Donc l'écriture {\color{red}1}.{\color{red}0}.{\color{red}1}.{\color{red}1}.{\color{red}0}.{\color{red}0}.{\color{red}1} représente l'entier : 
    $${\color{red}1} \times 64  + {\color{red}0} \times 32 + {\color{red}1} \times 16  + {\color{red}1} \times 8 + {\color{red}0} \times 4 + {\color{red}0} \times 2 + {\color{red}1} \times 1 = 64 + 16 + 8 + 1 = 89.$$
   \item \textbf{\'Ecriture binaire : formule.} 
 
 
  $$
\begin{array}{|c|c|c|c|c|c|c|c|}
  \hline
  {\color{red}b_{p-1}} & {\color{red}b_{p-2}}& {\color{red}\cdots} & {\color{red}b_i}& {\color{red}\cdots} & {\color{red}b_2} & {\color{red}b_1} & {\color{red}b_0} \\ 
  \hline\hline
  2^{p-1} & 2^{p-2} & \cdots & 2^i &\cdots & 2^2 & 2^1 & 2^0 \\
  \hline
\end{array}
$$   
  
On calcule donc l'entier correspondant aux \emph{bits} $[b_{p-1},b_{p-2}, \ldots,b_2,b_1,b_0]$ comme une somme de termes $b_i \times 2^i$, par la formule :
$$n = {\color{red}b_{p-1}} \times 2^{p-1} + {\color{red}b_{p-2}} \times 2^{p-2} + \cdots + {\color{red}b_i} 2^i +  \cdots + {\color{red}b_2} \times 2^2 + {\color{red}b_1} \times 2^1 + {\color{red}b_0} \times 2^0$$
  
    \item \textbf{\Python{} et le binaire.} \Python{} accepte que l'on écrive directement les entiers en écriture binaire, il suffit d'utiliser le préfixe \og{}\ci{0b}\fg{}.
    Exemples :
    \begin{itemize}
      \item avec \ci{x = 0b11010}, alors \ci{print(x)} affiche \ci{26},
      \item avec \ci{y = 0b11111}, alors \ci{print(y)} affiche \ci{31},
      \item et \ci{print(x+y)} affiche \ci{57}.
    \end{itemize} 
 \end{itemize}     

\end{cours}


%%%%%%%%%%%%%%%%%%%%%%%%%%%%%%%%%%%%%%%%%%%%%%%%%%%%%%%%%%%%%%%%
% Activité 2
%%%%%%%%%%%%%%%%%%%%%%%%%%%%%%%%%%%%%%%%%%%%%%%%%%%%%%%%%%%%%%%%


\begin{activite}[De l'écriture binaire vers l'entier]

\objectifs{Objectifs : à partir de l'écriture binaire, retrouver l'entier (en écriture décimale usuelle).}

\begin{enumerate}

  \item Calcule les entiers dont l'écriture binaire est donnée ci-dessous. Tu peux le faire à la main ou t'aider de \Python. Par exemple $1.0.0.1.1$ vaut $2^4+2^1+2^0 = 19$ ce que confirme la commande \ci{0b10011} qui renvoie $19$.
  
  \begin{itemize}
    \item $1.1$, $1.0.1$, $1.0.0.1$, $1.1.1.1$
    \item $1.0.0.0.0$, $1.0.1.0.1$, $1.1.1.1.1$
    \item $1.0.1.1.0.0$, $1.0.0.0.1.1$
    \item $1.1.1.0.0.1.0.1$
  \end{itemize}
  

  \item \'Ecris une fonction \ci{binaire_vers_entier(liste_binaire)} qui à partir d'une liste représentant l'écriture binaire calcule l'entier correspondant.
  
  \begin{fonction}[\ci{binaire_vers_entier()}]
  Usage : \ci{binaire_vers_entier(liste_binaire)} \\
  Entrée : une liste de \emph{bits} $0$ et $1$ \\
  Sortie : l'entier dont l'écriture binaire est la liste
  
  \medskip
  Exemples : 
  \begin{itemize}
    \item entrée \ci{[1,1,0]}, sortie \ci{6}
    \item entrée \ci{[1,1,0,1,1,1]}, sortie \ci{55}
    \item entrée \ci{[1,1,0,1,0,0,1,1,0,1,1,1]}, sortie \ci{3383}
  \end{itemize}       
  \end{fonction}
  
  \emph{Indications.}  Il faut cette fois  faire la somme d'éléments du type
    $$b_i \times 2^{i} \quad \text{ pour } 0 \le i < p$$
    où $p$ est la longueur de la liste et $b_i$ est le \emph{bit} ($0$ ou $1$) en position $i$ de la liste en comptant à partir de la fin.
  
  
  \item Voici un algorithme qui effectue le même travail : il permet le calcul de l'entier à partir de l'écriture binaire, mais il a l'avantage de ne jamais calculer directement des puissances de $2$. Programme cet algorithme en une fonction \ci{binaire_vers_entier_bis()} qui a les mêmes caractéristiques que la fonction précédente.
  
  \begin{algorithme}
  Entrée : \ci{liste} : une liste de $0$ et de $1$

  Sortie : le nombre binaire associé 

  \begin{itemize}
    \item  Initialiser une variable $n$ à $0$.

    
    \item  Pour chaque élément $b$ de \ci{liste} :
    
     \begin{itemize} 
       \item si $b$ vaut $0$, alors faire $n \leftarrow 2n$,
       \item si $b$ vaut $1$, alors faire $n \leftarrow 2n+1$.
     \end{itemize}    
         
    \item Le résultat est la valeur de $n$.
  \end{itemize} 
             
 \end{algorithme}
 
\end{enumerate}

\end{activite}
  
  
%%%%%%%%%%%%%%%%%%%%%%%%%%%%%%%%%%%%%%%%%%%%%%%%%%%%%%%%%%%%%%%%
%%%%%%%%%%%%%%%%%%%%%%%%%%%%%%%%%%%%%%%%%%%%%%%%%%%%%%%%%%%%%%%%

\begin{cours}[Écriture décimale (bis)]

Bien sûr, quand tu vois le nombre $1234$ tu sais tout de suite trouver la liste de ses  chiffres $[1,2,3,4]$. Mais comment faire en général à partir d'un entier $n$ ?
 
\begin{enumerate}
  \item On aura besoin de la division euclidienne par $10$ :
  \begin{itemize}
    \item on calcule par \ci{n \% 10} le \defi{reste}, on l'appelle aussi \defi{$n$ modulo $10$}
    \item on calcule par \ci{n // 10} le \defi{quotient} de cette division. 
  \end{itemize}
  
\myfigure{1}{
\tikzinput{fig-binaire-1}
}
  
  \item Les commandes \Python{} sont simplement \ci{n \% 10} et \ci{n // 10}.
  
  Exemple : \ci{1234 \% 10} vaut \ci{4} ; \ci{1234 // 10} vaut \ci{123}.
  
  \item
  \begin{itemize}
    \item Le chiffre des unités de $n$ s'obtient comme le reste modulo $10$ : c'est $n\%10$. Exemple $1234 \% 10 = 4$.
    \item Le chiffre des dizaines s'obtient à partir du quotient de la division de $n$ par $10$, puis en prenant le chiffre des unités de ce nombre : c'est donc $(n // 10) \% 10$.
   Exemple : $1234 // 10 = 123$, puis on a $123 \% 10 = 3$ ; le chiffre des dizaines de $1234$ est bien $3$.
   
   \item Pour le chiffre de centaines, on calcule le quotient de la division de $n$ par $100$, puis on prend le chiffre des unités.  Exemple $1234 // 100 = 12$ ; $2$ est le chiffre des unités de $12$ et c'est le chiffre des centaines de $1234$. Remarque : diviser par $100$, c'est diviser par $10$, puis de nouveau par $10$.
   
   \item Pour le chiffre des milliers on calcule le quotient de la division par $1000$ puis on prend le chiffre des unités\ldots
   \end{itemize} 
 \end{enumerate}     

\end{cours}



%%%%%%%%%%%%%%%%%%%%%%%%%%%%%%%%%%%%%%%%%%%%%%%%%%%%%%%%%%%%%%%%
% Activité 3
%%%%%%%%%%%%%%%%%%%%%%%%%%%%%%%%%%%%%%%%%%%%%%%%%%%%%%%%%%%%%%%%


\begin{activite}[Trouver les chiffres d'un entier]

\objectifs{Objectifs : décomposer un entier en la liste de ses chiffres (en base $10$).}

Programme l'algorithme suivant en une fonction \ci{entier_vers_decimale()}.

 
   \begin{fonction}[\ci{entier_vers_decimale()}]
  Usage : \ci{entier_vers_decimale(n)} \\
  Entrée : un entier positif \\
  Sortie : la liste de ses chiffres
  
  \medskip
    
  Exemple : si l'entrée est \ci{1234}, la sortie est \ci{[1,2,3,4]}.
  \end{fonction}


  \begin{algorithme}
  Entrée : un entier $n>0$

  Sortie : la liste de ses chiffres 

  \begin{itemize}
    \item Partir d'une liste vide.
    
    \item Tant que $n$ n'est pas nul :
    
     \begin{itemize} 
       \item ajouter $n \% 10$ au début de la liste,
       \item faire $n \leftarrow n//10$.
     \end{itemize}    
         
    \item Le résultat est la liste.
  \end{itemize} 
             
 \end{algorithme}
 
  

\end{activite}
%%%%%%%%%%%%%%%%%%%%%%%%%%%%%%%%%%%%%%%%%%%%%%%%%%%%%%%%%%%%%%%%
%%%%%%%%%%%%%%%%%%%%%%%%%%%%%%%%%%%%%%%%%%%%%%%%%%%%%%%%%%%%%%%%



\begin{cours}[\'Ecriture binaire avec \Python]

\Python{} calcule très bien l'écriture binaire d'un entier grâce à la fonction \ci{bin()}.
\index{bin@\ci{bin}}
  
   \begin{fonctionpython}[\ci{python : bin()}]
    Usage : \ci{bin(n)}\\
    Entrée : un entier  \\
    Sortie : l'écriture binaire de $n$ sous la forme d'une chaîne de caractères commençant par \ci{'0b'}
  
  \medskip
     
   Exemple :
  \begin{itemize}  
    \item \ci{bin(37)} renvoie \ci{'0b100101'}
    \item \ci{bin(139)} renvoie \ci{'0b10001011'}
  \end{itemize} 
  \end{fonctionpython}  
   
\end{cours}


\begin{cours}[Calcul de l'écriture binaire]

Pour calculer l'écriture binaire d'un entier $n$, c'est la même méthode que pour calculer l'écriture décimale mais en remplaçant les divisions par $10$ par des divisions par $2$.

Nous avons donc besoin :
  \begin{itemize}
    \item de $n\%2$ : le reste de la division euclidienne de $n$ par $2$ (appelé aussi $n$ modulo $2$) ; le reste vaut soit $0$ soit $1$.
    \item de $n//2$ : le quotient de cette division. 
  \end{itemize}
  
  Note que le reste  $n\%2$ vaut soit $0$ (quand $n$ est pair) soit $1$ (quand $n$ est impair).
  
  Voici la méthode générale pour calculer l'écriture binaire d'un entier :
\begin{itemize}
  \item On part de l'entier dont on veut l'écriture binaire.
  
  \item On effectue une suite de divisions euclidiennes par 2 : 
  \begin{itemize}
    \item à chaque division, on obtient un reste qui vaut 0 ou 1 ; 
    \item on obtient un quotient que l'on divise de nouveau par 2\ldots{} On s'arrête quand ce quotient est nul.
  \end{itemize}
  
  \item On lit l'écriture binaire comme la suite des restes, mais en partant du dernier reste.
\end{itemize}

\begin{exemple}
Calcul de l'écriture binaire de 14.

\begin{itemize}
  \item On divise 14 par 2, le quotient est 7, le reste est 0.
  \item On divise 7 (le quotient précédent) par 2 : le nouveau quotient est 3, le nouveau reste est 1.
  \item On divise 3 par 2 : quotient 1, reste 1.
  \item On divise 1 par 2 : quotient 0, reste 1.
  \item C'est terminé (le dernier quotient est nul).
  \item Les restes successifs sont 0, 1, 1, 1. On lit l'écriture binaire à l'envers c'est 1.1.1.0.  
\end{itemize}

Les divisions se font de gauche à droite, mais on lit les restes de droite à gauche.

\myfigure{1}{
\tikzinput{fig-binaire-2}
}
\end{exemple}

\begin{exemple}
Écriture binaire de 50.

\myfigure{1}{
\tikzinput{fig-binaire-3}
}

Les restes successifs sont 0, 1, 0, 0, 1, 1, donc l'écriture binaire de 50 est 1.1.0.0.1.0.
\end{exemple}


\end{cours}



%%%%%%%%%%%%%%%%%%%%%%%%%%%%%%%%%%%%%%%%%%%%%%%%%%%%%%%%%%%%%%%%
% Activité 4
%%%%%%%%%%%%%%%%%%%%%%%%%%%%%%%%%%%%%%%%%%%%%%%%%%%%%%%%%%%%%%%%


\begin{activite}[]

\objectifs{Objectifs : trouver l'écriture binaire d'un entier.}

\begin{enumerate}
  \item Calcule à la main l'écriture binaire des entiers suivants. Vérifie tes résultats à l'aide de la fonction \ci{bin()} de \Python.
  
   \begin{itemize}
    \item $13$, $18$, $29$, $31$,
    \item $44$, $48$, $63$, $64$,
    \item $100$, $135$, $239$, $1023$.
  \end{itemize} 
  
  \item
  Programme l'algorithme suivant en une fonction \ci{entier_vers_binaire()}.

  \begin{algorithme}
  Entrée : un entier $n>0$

  Sortie : son écriture binaire sous la forme d'une liste

  \begin{itemize}
    \item Partir d'une liste vide.
    
    \item Tant que $n$ n'est pas nul :
    
     \begin{itemize} 
       \item ajouter $n \% 2$ au début de la liste,
       \item faire $n \leftarrow n//2$.
     \end{itemize}    
         
    \item Le résultat est la liste.
  \end{itemize} 
             
 \end{algorithme}
 
   \begin{fonction}[\ci{entier_vers_binaire()}]
  Usage : \ci{entier_vers_binaire(n)} \\
  Entrée : un entier positif \\
  Sortie : son écriture binaire sous forme d'une liste
  
  \medskip  
  Exemple : si l'entrée est \ci{204}, la sortie est \ci{[1,1,0,0,1,1,0,0]}.
  \end{fonction} 
  
  Vérifie que tes fonctions marchent bien :
  \begin{itemize}
    \item pars d'un entier $n$,
    \item calcule son écriture binaire,
    \item calcule l'entier associé à cette écriture,
    \item tu dois retrouver l'entier de départ !
  \end{itemize}
\end{enumerate}
\end{activite}

\end{document}
